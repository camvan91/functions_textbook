In Exercises \ref{mappingfirst} - \ref{mappinglast}, determine whether or not the mapping diagram represents a function. Explain your reasoning. If the mapping does represent a function, state the domain, range, and represent the function as a set of ordered pairs.

\begin{multicols}{2}

\begin{enumerate}


\item  \label{mappingfirst} 

$~$

\begin{mfpic}[19]{-5}{5}{-5}{6}
\tlabel[cc](-4.25,5){Tennant}
\tlabel[cc](-4.25,3){Smith}
\tlabel[cc](-4.25,1){Calpadi}
\tlabel[cc](0,6){$M$}
\tlabel[cc](3.5,5){Eleven}
\tlabel[cc](3.5,3){Twelve}
\tlabel[cc](3.5,1){Thirteen}
\tlabel[cc](3.5,-1){Fourteen}
\arrow[l 5pt] \polyline{(-2.5, 5), (2.5, 5)}
\arrow[l 5pt] \polyline{(-2.5, 5), (2.5, 3)}
\arrow[l 5pt] \polyline{(-2.5, 3), (2.5, 1)}
\arrow[l 5pt] \polyline{(-2.5, 1), (2.5, -1)}
\end{mfpic} 



\item  \label{mappinglast} 

$~$

\begin{mfpic}[19]{-5}{5}{-5}{6}
\tlabel[cc](-4.25,5){Hartnell}
\tlabel[cc](-4.25,3){Cushing}
\tlabel[cc](-4.25,1){Hurndall}
\tlabel[cc](-4.25,-1){Troughton}
\tlabel[cc](0,6){$C$}
\tlabel[cc](3.5,5){One}
\tlabel[cc](3.5,3){Two}
\tlabel[cc](3.5,1){Three}
\arrow[l 5pt] \polyline{(-2.5, 5), (2.5, 5)}
\arrow[l 5pt] \polyline{(-2.5, 3), (2.5, 5)}
\arrow[l 5pt] \polyline{(-2.5, 1), (2.5, 5)}
\arrow[l 5pt] \polyline{(-2.5, -1), (2.5, 3)}
\end{mfpic}

\setcounter{HW}{\value{enumi}}

\end{enumerate}

\end{multicols}

In Exercises \ref{tablefirst} - \ref{tablelast}, determine whether or not the data in the given table represents $y$ as a function of $x$.  Explain your reasoning.  If the mapping does represent a function, state the domain, range, and represent the function as a set of ordered pairs.

\begin{multicols}{2}

\begin{enumerate}

\setcounter{enumi}{\value{HW}}

\item  \label{tablefirst} 

\[\begin{array}{|r||r|}  \hline

$x$  & \hphantom{\text{$-$}} $y$  \\ \hline
 -3 &  3 \\  \hline
 -2 & 2  \\  \hline
  -1 &  1  \\  \hline
 0 &  0 \\  \hline
 1 & 1  \\  \hline
 2 &  2 \\  \hline
 3 & 3  \\  \hline

\end{array}\]

\item \label{tablelast}

\[\begin{array}{|r||r|}  \hline

\hphantom{\text{$-$}} $x$  &$y$  \\ \hline

 0 & 0 \\  \hline
 1 & 1  \\  \hline
 1 & -1  \\  \hline
 2 &  2 \\  \hline
 2 & -2  \\  \hline
 3 &  3 \\  \hline
 3 & -3  \\  \hline

\end{array}\]

\setcounter{HW}{\value{enumi}}

\end{enumerate}

\end{multicols}

\begin{enumerate}

\setcounter{enumi}{\value{HW}}

\item    Suppose $W$ is the set of words in the English language and we set up a mapping from $W$ into the set of natural numbers $\mathbb{N}$ as follows: $\text{word} \rightarrow \text{number of letters in the word}$.  Explain why this mapping is a function.  What would you need to know to determine the range of the function?

\item  Suppose $L$ is the set of last names of all the people who have served or are currently serving as the President of the United States.   Consider the mapping from $L$ into $\mathbb{N}$ as follows:  $\text{last name} \rightarrow \text{number of their presidency}$.  For example,  $\text{Washington} \rightarrow 1$ and $\text{Obama} \rightarrow 44$.  Is this mapping a function?  What if we use full names instead of just last names? (\textbf{HINT:}  Research Grover Cleveland.)

\item  Under what conditions would the time of day be a function of the outdoor temperature?

\setcounter{HW}{\value{enumi}}

\end{enumerate}


For the functions $f$ described in Exercises \ref{buildfunctionfirst} - \ref{buildfunctionlast}, find $f(2)$ and find and simplify an expression for $f(x)$ that takes a real number $x$ and performs the following three steps in the order given: 


\begin{enumerate}
\setcounter{enumi}{\value{HW}}

\item  (1) multiply by 2; (2) add 3; (3) divide by 4. \label{buildfunctionfirst}

\item  (1) add 3; (2) multiply by 2; (3) divide by 4. 

\item (1) divide by 4; (2) add 3; (3) multiply by 2.

\item (1) multiply by 2; (2) add 3; (3) take the square root.

\item  (1) add 3; (2) multiply by 2; (3) take the square root.

\item  (1) add 3; (2) take the square root; (3) multiply by 2.  \label{buildfunctionlast}

\setcounter{HW}{\value{enumi}}
\end{enumerate}


In Exercises \ref{funcnotationbasicfirst} - \ref{funcnotationbasiclast}, use the given function $f$ to find and simplify the following:

\begin{multicols}{3}
\begin{itemize}
\item $f(3)$
\item $f(-1)$
\item $f\left(\frac{3}{2} \right)$
\end{itemize}
\end{multicols}

\begin{multicols}{3}
\begin{itemize}
\item  $f(4x)$
\item $4f(x)$
\item $f(-x)$
\end{itemize}
\end{multicols}

\begin{multicols}{3}
\begin{itemize}
\item  $f(x-4)$
\item $f(x) - 4$
\item  $f\left(x^2\right)$
\end{itemize}
\end{multicols}

\begin{multicols}{2}
\begin{enumerate}
\setcounter{enumi}{\value{HW}}

\item  $f(x) = 2x+1$ \label{funcnotationbasicfirst} 
\item  $f(x) = 3 - 4x$

\setcounter{HW}{\value{enumi}}
\end{enumerate}
\end{multicols}

\begin{multicols}{2}
\begin{enumerate}
\setcounter{enumi}{\value{HW}}

\item $f(x) = 2 - x^2$
\item $f(x) = x^2 - 3x + 2$

\setcounter{HW}{\value{enumi}}
\end{enumerate}
\end{multicols}


\begin{multicols}{2}
\begin{enumerate}
\setcounter{enumi}{\value{HW}}

\item $f(x) = 6$
\item $f(x) = 0$ \label{funcnotationbasiclast}

\setcounter{HW}{\value{enumi}}
\end{enumerate}
\end{multicols}

In Exercises \ref{secondfuncnotationbasicfirst} - \ref{secondfuncnotationbasiclast}, use the given function $f$ to find and simplify the following:

\begin{multicols}{3}
\begin{itemize}

\item  $f(2)$
\item  $f(-2)$
\item  $f(2a)$

\end{itemize}
\end{multicols}

\begin{multicols}{3}
\begin{itemize}

\item  $2 f(a)$
\item $f(a+2)$
\item $f(a) + f(2)$

\end{itemize}
\end{multicols}

\begin{multicols}{3}
\begin{itemize}

\item  $f \left( \dfrac{2}{a} \right)$
\item $\dfrac{f(a)}{2}$
\item  $f(a + h)$

\end{itemize}
\end{multicols}


\begin{multicols}{2}
\begin{enumerate}
\setcounter{enumi}{\value{HW}}

\item $f(x) = 2x-5$ \label{secondfuncnotationbasicfirst}
\item $f(t) = 5-2t$

\setcounter{HW}{\value{enumi}}
\end{enumerate}
\end{multicols}

\begin{multicols}{2}
\begin{enumerate}
\setcounter{enumi}{\value{HW}}

\item $f(w) = 2w^2 - 1$
\item $f(q) = 3q^2+3q-2$

\setcounter{HW}{\value{enumi}}
\end{enumerate}
\end{multicols}
 
\begin{multicols}{2}
\begin{enumerate}
\setcounter{enumi}{\value{HW}}


\item $f(r) = 117$
\item $f(z) = \dfrac{z}{2}$  \label{secondfuncnotationbasiclast}

\setcounter{HW}{\value{enumi}}
\end{enumerate}
\end{multicols}


\newpage


In Exercises \ref{findzerofuncfirst} - \ref{findzerofunclast}, use the given function $f$ to find $f(0)$ and solve $f(x) = 0$

\begin{multicols}{2}
\begin{enumerate}
\setcounter{enumi}{\value{HW}}

\item $f(x) = 2x - 1$ \label{findzerofuncfirst}
\item $f(x) = 3 - \frac{2}{5} x$

\setcounter{HW}{\value{enumi}}
\end{enumerate}
\end{multicols}

\begin{multicols}{2}
\begin{enumerate}
\setcounter{enumi}{\value{HW}}

\item $f(x) = 2x^2 - 6$
\item $f(x) = x^2 - x - 12$ \label{findzerofunclast}

\setcounter{HW}{\value{enumi}}
\end{enumerate}
\end{multicols}


In Exercises \ref{equfunctionfirst} - \ref{equfunctionlast}, determine whether or not the equation represents $y$ as a function of $x$.  



\begin{multicols}{3}
\begin{enumerate}
\setcounter{enumi}{\value{HW}}

\item $y = x^{3} - x$ \label{equfunctionfirst}
\item $y = \sqrt{x - 2}$
\item $x^{3}y = -4$ 
\setcounter{HW}{\value{enumi}}
\end{enumerate}
\end{multicols}

\begin{multicols}{3}
\begin{enumerate}
\setcounter{enumi}{\value{HW}}

\item $x^{2} - y^{2} = 1$
\item $y = \dfrac{x}{x^{2} - 9}$
\item $x = -6$

\setcounter{HW}{\value{enumi}}
\end{enumerate}
\end{multicols}

\begin{multicols}{3}
\begin{enumerate}
\setcounter{enumi}{\value{HW}}

\item  $x = y^2 + 4$

\item $y = x^2 + 4$
\item $x^2 + y^2 = 4$

\setcounter{HW}{\value{enumi}}
\end{enumerate}
\end{multicols}

\begin{multicols}{3}
\begin{enumerate}
\setcounter{enumi}{\value{HW}}


\item $y = \sqrt{4-x^2}$
\item $x^2 - y^2 = 4$
\item $x^3 + y^3 = 4$


\setcounter{HW}{\value{enumi}}
\end{enumerate}
\end{multicols}

\begin{multicols}{3}
\begin{enumerate}
\setcounter{enumi}{\value{HW}}

\item $2x + 3y = 4$
\item $2xy = 4$
\item $x^2 = y^2$ \label{equfunctionlast}

\setcounter{HW}{\value{enumi}}
\end{enumerate}
\end{multicols}





Exercises \ref{setfunctionfirst} - \ref{setfunctionlast} give a set of points in the $xy$-plane.    Determine if $y$ is a function of $x$.  If so, state the domain and range.

\begin{enumerate}

\setcounter{enumi}{\value{HW}}

\item \{$(-3, 9)$, $\;(-2, 4)$, $\;(-1, 1)$, $\;(0, 0)$, $\;(1, 1)$, $\;(2, 4)$, $\;(3, 9)\}$ \label{setfunctionfirst}
\item  $\left\{ (-3,0), (1,6), (2, -3), (4,2), (-5,6), (4, -9), (6,2) \right\}$
\item  $\left\{ (-3,0), (-7,6), (5,5), (6,4), (4,9), (3,0) \right\}$
\item  $\left\{ (1,2), (4,4), (9,6), (16,8), (25,10), (36, 12), \ldots \right\}$
\item \{($x, y) \, | \, x$ is an odd integer, and $y$ is an even integer\}
\item \{$(x, 1) \, | \, x$ is an irrational number\}

\item $\{ (1,0), (2,1), (4,2), (8,3), (16,4), (32, 5), \ldots \}$
\item $\{ \ldots (-3,9), (-2,4), (-1,1), (0,0), (1,1), (2,4), (3,9), \ldots \}$

\setcounter{HW}{\value{enumi}}
\end{enumerate}

\vspace{-0.1in}

\begin{multicols}{2}
\begin{enumerate}
\setcounter{enumi}{\value{HW}}

\item $\{ (-2, y) \, | \, -3 < y < 4\}$
\item  $\{ (x,3) \, | \,  -2 \leq x < 4\}$

\setcounter{HW}{\value{enumi}}
\end{enumerate}
\end{multicols}

\begin{multicols}{2}
\begin{enumerate}
\setcounter{enumi}{\value{HW}}


\item  $\{ \left(x,x^2\right) \, | \, \text{$x$ is a real number} \}$
\item  $\{ \left(x^2,x\right) \, | \, \text{$x$ is a real number} \}$ \label{setfunctionlast}

\setcounter{HW}{\value{enumi}}
\end{enumerate}
\end{multicols}

\begin{enumerate}
\setcounter{enumi}{\value{HW}}

\item  \label{HLTExercise} The Vertical Line Test is a quick way to determine from a graph if the vertical axis variable is a function of the horizontal axis variable. If we are given a graph and asked to determine if the horizontal axis variable is a function of the vertical axis variable, we can use horizontal lines instead of vertical lines to check.  Using Theorem \ref{VLT} as a guide,  formulate a `Horizontal Line Test.'  (We'll refer back to this exercise in Section \ref{InverseFunctions}.)

\setcounter{HW}{\value{enumi}}
\end{enumerate}

\newpage


In Exercises \ref{graphfunctionfirstxy} - \ref{graphfunctionlastxy}, determine whether or not the graph suggests $y$ is a function of $x$.  For the ones which do, state the domain and range. 

\begin{multicols}{2}
\begin{enumerate}
\setcounter{enumi}{\value{HW}}

\item $~$ \vspace{-.1in} \label{graphfunctionfirstxy}

\begin{mfpic}[17]{-5}{2}{-2}{5}
\point[4pt]{(-4, -1), (-3, 0), (-2, 1), (-1, 2), (0, 3), (1, 4)}
\axes
\tlabel[cc](2,-0.5){\scriptsize $x$}
\tlabel[cc](0.5,4.75){\scriptsize $y$}
\xmarks{-4,-3,-2,-1,1}
\ymarks{-1,1,2,3,4}
\tlpointsep{4pt}
\axislabels {x}{{\tiny $-4 \hspace{8pt}$} -4, {\tiny $-3 \hspace{8pt}$} -3, {\tiny $-2 \hspace{8pt}$} -2, {\tiny $-1 \hspace{8pt}$} -1, {\tiny $1$} 1}
\axislabels {y}{{\tiny $-1$} -1, {\tiny $1$} 1, {\tiny $2$} 2, {\tiny $3$} 3, {\tiny $4$} 4}
\end{mfpic}

\vfill
\columnbreak

\item $~$  \label{graphfunctionfirstxy2}

\begin{mfpic}[15]{-5}{2}{-2}{5}
\point[4pt]{(-4, -1), (-3, 0), (-3, 1), (-2, 1), (-1, 2), (0, 3), (1, 4)}
\axes
\tlabel[cc](2,-0.5){\scriptsize $x$}
\tlabel[cc](0.5,4.75){\scriptsize $y$}
\xmarks{-4,-3,-2,-1,1}
\ymarks{-1,1,2,3,4}
\tlpointsep{4pt}
\axislabels {x}{{\tiny $-4 \hspace{6pt}$} -4, {\tiny $-3 \hspace{6pt}$} -3, {\tiny $-2 \hspace{6pt}$} -2, {\tiny $-1 \hspace{6pt}$} -1, {\tiny $1$} 1}
\axislabels {y}{{\tiny $-1$} -1, {\tiny $1$} 1, {\tiny $2$} 2, {\tiny $3$} 3, {\tiny $4$} 4}
\end{mfpic}


\setcounter{HW}{\value{enumi}}
\end{enumerate}
\end{multicols}


\begin{multicols}{2}
\begin{enumerate}
\setcounter{enumi}{\value{HW}}



\item $~$ \label{graphfunctionfirstxy3}

\begin{mfpic}[15]{-3}{3}{-1}{6}
\axes
\tlabel[cc](3,-0.5){\scriptsize $x$}
\tlabel[cc](0.5,5.75){\scriptsize $y$}
\xmarks{-2,-1,1,2}
\ymarks{1,2,3,4,5}
\tlpointsep{4pt}
\axislabels {x}{{\tiny $-2 \hspace{8pt}$} -2, {\tiny $-1 \hspace{8pt}$} -1, {\tiny $1$} 1, {\tiny $2$} 2}
\axislabels {y}{{\tiny $1$} 1, {\tiny $2$} 2, {\tiny $3$} 3, {\tiny $4$} 4, {\tiny $5$} 5}
\penwd{1.25pt}
\arrow \reverse \arrow \function{-2.1, 2.1, 0.1}{x**2+1}
\end{mfpic}

\vfill
\columnbreak

\item $~$  \label{graphfunctionlastxy}

\begin{mfpic}[15]{-4}{4}{-4}{4}
\axes
\tlabel[cc](4,-0.5){\scriptsize $x$}
\tlabel[cc](0.5,3.75){\scriptsize $y$}
\xmarks{-3,-2,-1,1,2,3}
\ymarks{-3,-2,-1,1,2,3}
\tlpointsep{4pt}
\axislabels {x}{{\tiny $-3 \hspace{8pt}$} -3, {\tiny $-2 \hspace{8pt}$} -2, {\tiny $-1 \hspace{8pt}$} -1, {\tiny $1$} 1, {\tiny $2$} 2, {\tiny $3$} 3}
\axislabels {y}{{\tiny $-3$} -3, {\tiny $-2$} -2, {\tiny $-1$} -1, {\tiny $1$} 1, {\tiny $2$} 2, {\tiny $3$} 3}
\penwd{1.25pt}
\arrow \reverse \arrow \parafcn{-2,2,0.1}{(cosh(t),sinh(t))}
\end{mfpic}


\setcounter{HW}{\value{enumi}}
\end{enumerate}
\end{multicols}

\begin{enumerate}
\setcounter{enumi}{\value{HW}}

\item   Determine which, if any, of the graphs in numbers \ref{graphfunctionfirstxy} - \ref{graphfunctionlastxy} represent $x$ as a function of $y$.  For the ones which do, state the domain and range.  (Feel free to use Exercise \ref{HLTExercise}.)

\setcounter{HW}{\value{enumi}}
\end{enumerate}

In Exercises \ref{graphfunctionfirstvw} - \ref{graphfunctionlastvw}, determine whether or not the graph suggests $w$ is a function of $v$ .  For the ones which do, state the domain and range. 


\begin{multicols}{2}
\begin{enumerate}
\setcounter{enumi}{\value{HW}}

\item $~$ \label{graphfunctionfirstvw}  

\begin{mfpic}[15]{-1}{10}{-1}{4}
\axes
\tlabel[cc](10,-0.5){\scriptsize $v$}
\tlabel[cc](0.5,3.75){\scriptsize $w$}
\xmarks{1,2,3,4,5,6,7,8,9}
\ymarks{1,2,3}
\tlpointsep{4pt}
\axislabels {x}{{\tiny $1$} 1, {\tiny $2$} 2, {\tiny $3$} 3, {\tiny $4$} 4, {\tiny $5$} 5, {\tiny $6$} 6, {\tiny $7$} 7, {\tiny $8$} 8, {\tiny $9$} 9}
\axislabels {y}{{\tiny $1$} 1, {\tiny $2$} 2, {\tiny $3$} 3}
\penwd{1.25pt}
\arrow \function{2, 10, 0.1}{sqrt(x - 2)}
\point[4pt]{(2,0)}
\end{mfpic}

\vfill
\columnbreak

\item $~$ \label{graphfunctionfirstvw2} 

\begin{mfpic}[15]{-5}{5}{-1}{5}
\axes
\tlabel[cc](5,-0.5){\scriptsize $v$}
\tlabel[cc](0.5,4.75){\scriptsize $w$}
\xmarks{-4,-3,-2,-1,1,2,3,4}
\ymarks{1,2,3,4}
\tlpointsep{4pt}
\axislabels {x}{{\tiny $-4 \hspace{8pt}$} -4, {\tiny $-3 \hspace{8pt}$} -3, {\tiny $-2 \hspace{8pt}$} -2, {\tiny $-1 \hspace{8pt}$} -1, {\tiny $1$} 1, {\tiny $2$} 2, {\tiny $3$} 3, {\tiny $4$} 4}
\axislabels {y}{{\tiny $1$} 1, {\tiny $2$} 2, {\tiny $3$} 3, {\tiny $4$} 4}
\penwd{1.25pt}
\arrow \reverse \arrow \function{-5, 5, 0.1}{4/(x**2 + 1)}
\end{mfpic}


\setcounter{HW}{\value{enumi}}
\end{enumerate}
\end{multicols}

\newpage

\begin{multicols}{2}
\begin{enumerate}
\setcounter{enumi}{\value{HW}}

\item $~$  \label{graphfunctionfirstvw3} 

\begin{mfpic}[15]{-4.5}{5.5}{-4}{3}
\axes
%\fillcolor[gray]{.7}
%\gfill \rect{(-3.97, -2.97), (4.97, 1.97)}
%\dashed \polyline{(-4, 2), (5, 2)}
%\dashed \polyline{(5, 2), (5, -3)}
%\dashed \polyline{(5, -3), (-4, -3)}
\tlabel[cc](5.5,-0.5){\scriptsize $v$}
\tlabel[cc](0.5,2.75){\scriptsize $w$}
\xmarks{-4,-3,-2,-1,1,2,3,4,5}
\ymarks{-3,-2,-1,1,2}
\tlpointsep{4pt}
\axislabels {x}{{\tiny $-4 \hspace{8pt}$} -4,  {\tiny $-2 \hspace{8pt}$} -2, {\tiny $-1 \hspace{8pt}$} -1, {\tiny $1$} 1, {\tiny $2$} 2, {\tiny $4$} 4, {\tiny $5$} 5}
\axislabels {y}{ {\tiny $-2$} -2, {\tiny $-1$} -1, {\tiny $1$} 1}
\penwd{1.25pt}
\polyline{(-3, -3), (-3, 2), (3,2), (3,-3), (-3,-3)}
\end{mfpic}



\item $~$ \label{graphfunctionlastvw}

\begin{mfpic}[15]{-6}{4}{-2.5}{4.5}
\axes
\tlabel[cc](4,-0.5){\scriptsize $v$}
\tlabel[cc](0.5,4.75){\scriptsize $w$}
\xmarks{-5 step 1 until 3}
\ymarks{-2 step 1 until 4}
\tlpointsep{4pt}
\axislabels {x}{{\tiny $-5 \hspace{8pt}$} -5, {\tiny $-4 \hspace{8pt}$} -4, {\tiny $-3 \hspace{8pt}$} -3, {\tiny $-2 \hspace{8pt}$} -2, {\tiny $-1 \hspace{8pt}$} -1, {\tiny $1$} 1, {\tiny $2$} 2, {\tiny $3$} 3}
\axislabels {y}{{\tiny $-2$} -2, {\tiny $-1$} -1, {\tiny $1$} 1, {\tiny $2$} 2, {\tiny $3$} 3, {\tiny $4$} 4}
\penwd{1.25pt}
\function{-5,-1,0.1}{-5 - 6*x - x**2}
\function{-1,3,0.1}{x/4 - 7/4}
\point[4pt]{(-5, 0), (-1, 0)}
\pointfillfalse
\point[4pt]{(-3,4), (-1,-2), (3,-1)}
\end{mfpic}


\setcounter{HW}{\value{enumi}}
\end{enumerate}
\end{multicols}

\begin{enumerate}
\setcounter{enumi}{\value{HW}}

\item  Determine which, if any, of the graphs in numbers \ref{graphfunctionfirstvw} - \ref{graphfunctionlastvw} represent $v$ as a function of $w$.   For the ones which do, state the domain and range.  (Feel free to use Exercise \ref{HLTExercise}.)

\setcounter{HW}{\value{enumi}}
\end{enumerate}

In Exercises \ref{graphfunctionfirsttT} - \ref{graphfunctionlasttT}, determine whether or not the graph suggests $T$ is a function of $t$.   For the ones which do, state the domain and range. 


\begin{multicols}{2}
\begin{enumerate}
\setcounter{enumi}{\value{HW}}

\item  $~$ \label{graphfunctionfirsttT}

\begin{mfpic}[8]{-4}{4}{-6}{10}

\axes
\tlabel[cc](4,-0.5){\scriptsize $t$}
\tlabel[cc](0.5,10){\scriptsize $T$}
\xmarks{-3,-2,-1,1,2,3}
\ymarks{-5,-4,-3,-2,-1,1,2,3,4,5,6,7,8,9}
\tlpointsep{4pt}
\axislabels {x}{{\tiny $-3 \hspace{6pt}$} -3,{\tiny $-2 \hspace{6pt}$} -2, {\tiny $-1 \hspace{6pt}$} -1, {\tiny $1$} 1, {\tiny $2$} 2, {\tiny $3$} 3}
\axislabels {y}{{\tiny $-5$} -5, {\tiny $-4$} -4, {\tiny $-3$} -3, {\tiny $-2$} -2, {\tiny $-1$} -1, {\tiny $1$} 1, {\tiny $2$} 2, {\tiny $3$} 3, {\tiny $4$} 4, {\tiny $5$} 5, {\tiny $6$} 6, {\tiny $7$} 7, {\tiny $8$} 8, {\tiny $9$} 9}
\penwd{1.25pt}
\arrow \function{-2,4.5,0.1}{x**2 - 2*x - 2}
\point[4pt]{(-2,6), (1,-3) }
\end{mfpic}

\vfill
\columnbreak

\item  $~$  \label{graphfunctionfirsttT2}

\begin{mfpic}[10]{-6}{6}{-6}{6}
\axes
\tlabel[cc](6,-0.5){\scriptsize $t$}
\tlabel[cc](0.5,6){\scriptsize $T$}
\xmarks{-5,-4,-3,-2,-1,1,2,3,4,5}
\ymarks{-5,-4,-3,-2,-1,1,2,3,4,5}
\tlpointsep{4pt}
\axislabels {x}{{\tiny $-5 \hspace{6pt}$} -5,{\tiny $-4 \hspace{6pt}$} -4,{\tiny $-3 \hspace{6pt}$} -3,{\tiny $-2 \hspace{6pt}$} -2, {\tiny $-1 \hspace{6pt}$} -1, {\tiny $1$} 1, {\tiny $2$} 2, {\tiny $3$} 3, {\tiny $4$} 4, {\tiny $5$} 5}
\penwd{1.25pt}
\plrfcn{0,180,5}{5*sind 3t}
\end{mfpic} 


\setcounter{HW}{\value{enumi}}
\end{enumerate}
\end{multicols}

\begin{multicols}{2}
\begin{enumerate}
\setcounter{enumi}{\value{HW}}

\item  $~$  \label{graphfunctionfirsttT3}

\begin{mfpic}[10]{-6}{6}{-6}{6}
\axes
\tlabel[cc](6,-0.5){\scriptsize $t$}
\tlabel[cc](0.5,6){\scriptsize $T$}
\xmarks{-5,-4,-3,-2,-1,1,2,3,4,5}
\ymarks{-5,-4,-3,-2,-1,1,2,3,4,5}
\tlpointsep{4pt}
\axislabels {x}{{\tiny $-5 \hspace{6pt}$} -5,{\tiny $-4 \hspace{6pt}$} -4,{\tiny $-3 \hspace{6pt}$} -3,{\tiny $-2 \hspace{6pt}$} -2, {\tiny $-1 \hspace{6pt}$} -1, {\tiny $1$} 1, {\tiny $2$} 2, {\tiny $3$} 3, {\tiny $4$} 4, {\tiny $5$} 5}
\axislabels {y}{{\tiny $-5$} -5,{\tiny $-4$} -4,{\tiny $-3$} -3, {\tiny $-2$} -2, {\tiny $-1$} -1, {\tiny $1$} 1, {\tiny $2$} 2, {\tiny $3$} 3, {\tiny $4$} 4, {\tiny $5$} 5}
\penwd{1.25pt}
\function{-5,4,0.1}{0.0502*(x**3) - 0.0344*(x**2) - 0.2010*x + 2.138}
\pointfillfalse
\point[4pt]{(-5,-4),(4,4)}

\end{mfpic} 

\vfill
\columnbreak

\item  $~$ \label{graphfunctionlasttT}

\begin{mfpic}[10]{-2}{7}{-6}{6}
\axes
\tlabel[cc](7,-0.5){\scriptsize $t$}
\tlabel[cc](0.5,6){\scriptsize $T$}
\xmarks{-1,1,2,3,4,5,6}
\ymarks{-5,-4,-3,-2,-1,1,2,3,4,5}
\tlpointsep{4pt}
\axislabels {x}{{\tiny $-1 \hspace{6pt}$} -1, {\tiny $1$} 1, {\tiny $2$} 2, {\tiny $3$} 3, {\tiny $4$} 4, {\tiny $5$} 5, {\tiny $6$} 6}
\axislabels {y}{{\tiny $-5$} -5,{\tiny $-4$} -4,{\tiny $-3$} -3, {\tiny $-2$} -2, {\tiny $-1$} -1, {\tiny $1$} 1, {\tiny $2$} 2, {\tiny $3$} 3, {\tiny $4$} 4, {\tiny $5$} 5}
\penwd{1.25pt}
\polyline{(0,-1), (3,-4)}
\polyline{(3,1), (4,4), (6,0)}
\point[4pt]{(0,-1), (4,4), (6,0)}
\pointfillfalse
\point[4pt]{(3,-4), (3,1)}
\end{mfpic} 

\setcounter{HW}{\value{enumi}}
\end{enumerate}
\end{multicols}


\begin{enumerate}
\setcounter{enumi}{\value{HW}}

\item  Determine which, if any, of the graphs in numbers \ref{graphfunctionfirsttT} - \ref{graphfunctionlasttT} represent $t$ as a function of $T$.  For the ones which do, state the domain and range.   (Feel free to use Exercise \ref{HLTExercise}.)

\setcounter{HW}{\value{enumi}}
\end{enumerate}


In Exercises \ref{graphfunctionfirstHs} - \ref{graphfunctionlastHs}, determine whether or not the graph suggests $H$ is a function of $s$.  For the ones which do, state the domain and range. 


\begin{multicols}{2}
\begin{enumerate}
\setcounter{enumi}{\value{HW}}

\item  $~$ \label{graphfunctionfirstHs}

\begin{mfpic}[15]{-3}{3}{-1}{5}
\axes
\tlabel[cc](3,-0.5){\scriptsize $s$}
\tlabel[cc](0.5,5){\scriptsize $H$}
\xmarks{-2,-1,1,2}
\ymarks{1,2,3,4}
\tlpointsep{4pt}
\axislabels {x}{{\tiny $-2 \hspace{6pt}$} -2, {\tiny $-1 \hspace{6pt}$} -1, {\tiny $1$} 1, {\tiny $2$} 2}
\axislabels {y}{{\tiny $1$} 1, {\tiny $2$} 2, {\tiny $3$} 3, {\tiny $4$} 4}
\penwd{1.25pt}
\arrow \reverse \arrow \function{-2.25,2.25,0.1}{4-(x**2)}
\end{mfpic} 

\vfill
\columnbreak

\item  $~$  \label{graphfunctionfirstHs2}


\begin{mfpic}[15]{-3}{3}{-1}{5}
\axes
\tlabel[cc](3,-0.5){\scriptsize $s$}
\tlabel[cc](0.5,5){\scriptsize $H$}
\xmarks{-2,-1,1,2}
\ymarks{1,2,3,4}
\tlpointsep{4pt}
\axislabels {x}{{\tiny $-2 \hspace{6pt}$} -2, {\tiny $-1 \hspace{6pt}$} -1, {\tiny $1$} 1, {\tiny $2$} 2}
\axislabels {y}{{\tiny $1$} 1, {\tiny $2$} 2, {\tiny $3$} 3, {\tiny $4$} 4}
\penwd{1.25pt}
\arrow \reverse \arrow \polyline{(-2,-1), (1,4), (2,-1)}
\end{mfpic} 

\setcounter{HW}{\value{enumi}}
\end{enumerate}
\end{multicols}



\begin{multicols}{2}
\begin{enumerate}
\setcounter{enumi}{\value{HW}}

\item  $~$  \label{graphfunctionfirstHs3}

\begin{mfpic}[15]{-3}{3}{-1}{5}
\axes
\tlabel[cc](3,-0.5){\scriptsize $s$}
\tlabel[cc](0.5,5){\scriptsize $H$}
\xmarks{-2,-1,1,2}
\ymarks{1,2,3,4}
\tlpointsep{4pt}
\axislabels {x}{{\tiny $-2 \hspace{6pt}$} -2, {\tiny $-1 \hspace{6pt}$} -1, {\tiny $1$} 1, {\tiny $2$} 2}
\axislabels {y}{{\tiny $1$} 1, {\tiny $2$} 2, {\tiny $3$} 3, {\tiny $4$} 4}
\penwd{1.25pt}
\arrow \function{-2, 1.8, 0.1}{3-2*sqrt(x+2)}
\point[4pt]{(-2,3)}
\end{mfpic} 

\vfill

\item  $~$ \label{graphfunctionlastHs}


\begin{mfpic}[15]{-3}{3}{-1}{5}
\axes
\tlabel[cc](3,-0.5){\scriptsize $s$}
\tlabel[cc](0.5,5){\scriptsize $H$}
\xmarks{-2,-1,1,2}
\ymarks{1,2,3,4}
\tlpointsep{4pt}
\axislabels {x}{{\tiny $-2 \hspace{6pt}$} -2, {\tiny $-1 \hspace{6pt}$} -1, {\tiny $1$} 1, {\tiny $2$} 2}
\axislabels {y}{{\tiny $1$} 1, {\tiny $2$} 2, {\tiny $3$} 3, {\tiny $4$} 4}
\penwd{1.25pt}
\arrow \reverse \arrow \function{-2.15, 1.75, 0.1}{x*(x-1)*(x+2)}
\end{mfpic} 

\setcounter{HW}{\value{enumi}}
\end{enumerate}
\end{multicols}

\begin{enumerate}
\setcounter{enumi}{\value{HW}}

\item  Determine which, if any, of the graphs in numbers \ref{graphfunctionfirstHs} - \ref{graphfunctionlastHs} represent $s$ as a function of $H$.  For the ones which do, state the domain and range.   (Feel free to use Exercise \ref{HLTExercise}.)

\setcounter{HW}{\value{enumi}}
\end{enumerate}


In Exercises \ref{graphfunctionfirstut} - \ref{graphfunctionlastut}, determine whether or not the graph suggests $u$ is a function of $t$. For the ones which do, state the domain and range. 


\begin{multicols}{2}
\begin{enumerate}
\setcounter{enumi}{\value{HW}}

\item  $~$   \label{graphfunctionfirstut}

\begin{mfpic}[15]{-3}{3}{-3}{3}
\axes
\tlabel[cc](3,-0.5){\scriptsize $t$}
\tlabel[cc](0.5,3){\scriptsize $u$}
\xmarks{-2,-1,1,2}
\ymarks{-2,-1,1,2}
\tlpointsep{4pt}
\axislabels {x}{{\tiny $-2 \hspace{6pt}$} -2, {\tiny $-1 \hspace{6pt}$} -1, {\tiny $1$} 1, {\tiny $2$} 2}
\axislabels {y}{{\tiny $1$} 1, {\tiny $2$} 2, {\tiny $-2$} -2, {\tiny $-1$} -1}
\penwd{1.25pt}
\arrow \polyline{(0,1), (-2,-2)}
\arrow \polyline{(1,2), (3,2)}
\point[4pt]{(0,1)}
\pointfillfalse
\point[4pt]{(1,2)}
\end{mfpic} 

\vfill
\columnbreak

\item  $~$  \label{graphfunctionfirstut2}


\begin{mfpic}[15]{-4}{4}{-3}{3}
\axes
\tlabel[cc](4,-0.5){\scriptsize $t$}
\tlabel[cc](0.5,3){\scriptsize $u$}
\xmarks{-3,-2,-1,1,2,3}
\ymarks{-2,-1,1,2}
\tlpointsep{4pt}
\axislabels {x}{{\tiny $-3 \hspace{6pt}$} -3,{\tiny $-2 \hspace{6pt}$} -2, {\tiny $-1 \hspace{6pt}$} -1, {\tiny $1$} 1, {\tiny $2$} 2, {\tiny $3$} 3}
\axislabels {y}{{\tiny $1$} 1, {\tiny $2$} 2, {\tiny $-2$} -2, {\tiny $-1$} -1}
\penwd{1.25pt}
\function{-3,3,0.1}{2*sin(1.05*x)}
\point[4pt]{(-3,0)}
\point[4pt]{(3,0)}
\end{mfpic} 

\setcounter{HW}{\value{enumi}}
\end{enumerate}
\end{multicols}

\newpage

\begin{multicols}{2}
\begin{enumerate}
\setcounter{enumi}{\value{HW}}

\item  $~$  \label{graphfunctionfirstut3}

\begin{mfpic}[15]{-3}{3}{-3}{3}
\axes
\tlabel[cc](3,-0.5){\scriptsize $t$}
\tlabel[cc](0.5,3){\scriptsize $u$}
\xmarks{-2,-1,1,2}
\ymarks{-2,-1,1,2}
\tlpointsep{4pt}
\axislabels {x}{{\tiny $-2 \hspace{6pt}$} -2, {\tiny $-1 \hspace{6pt}$} -1, {\tiny $1$} 1, {\tiny $2$} 2}
\axislabels {y}{{\tiny $1$} 1, {\tiny $2$} 2, {\tiny $-2$} -2, {\tiny $-1$} -1}
\penwd{1.25pt}
\arrow \reverse \arrow \polyline{(2,-3), (2,3)}
\end{mfpic} 

\vfill
\columnbreak

\item  $~$ \label{graphfunctionlastut}


\begin{mfpic}[15]{-3}{3}{-3}{3}
\axes
\tlabel[cc](3,-0.5){\scriptsize $t$}
\tlabel[cc](0.5,3){\scriptsize $u$}
\xmarks{-2,-1,1,2}
\ymarks{-2,-1,1,2}
\tlpointsep{4pt}
\axislabels {x}{{\tiny $-2 \hspace{6pt}$} -2, {\tiny $-1 \hspace{6pt}$} -1, {\tiny $1$} 1, {\tiny $2$} 2}
\axislabels {y}{{\tiny $1$} 1, {\tiny $2$} 2, {\tiny $-2$} -2, {\tiny $-1$} -1}
\penwd{1.25pt}
\arrow \reverse \arrow \polyline{(-3,2), (3,2)}
\end{mfpic} 

\setcounter{HW}{\value{enumi}}
\end{enumerate}
\end{multicols}

\begin{enumerate}
\setcounter{enumi}{\value{HW}}

\item  Determine which, if any, of the graphs in numbers \ref{graphfunctionfirstut} - \ref{graphfunctionlastut} represent $t$ as a function of $u$.  For the ones which do, state the domain and range.   (Feel free to use Exercise \ref{HLTExercise}.)

\setcounter{HW}{\value{enumi}}
\end{enumerate}



In Exercises \ref{functionvaluesfromgraphfirst} - \ref{functionvaluesfromgraphlast}, use the graphs of $f$ and $g$ below to find the indicated values.

\begin{multicols}{2}

\begin{mfpic}[15]{-6}{6}{-6}{6}

\axes
\tlabel[cc](6,-0.5){\scriptsize $x$}
\tlabel[cc](0.5,6){\scriptsize $y$}
\xmarks{-5,-4,-3,-2,-1,1,2,3,4,5}
\ymarks{-5,-4,-3,-2,-1,1,2,3,4,5}
\tlpointsep{5pt}
\scriptsize
\axislabels {x}{{$-5 \hspace{7pt}$} -5,{$-4 \hspace{7pt}$} -4,{$-3 \hspace{7pt}$} -3,{$-2 \hspace{7pt}$} -2, {$-1 \hspace{7pt}$} -1, {$1$} 1, {$2$} 2, {$3$} 3, {$4$} 4, {$5$} 5}
\axislabels {y}{{$-5$} -5,{$-4$} -4,{$-3$} -3,{$-2$} -2,{$-1$} -1, {$1$} 1, {$2$} 2, {$3$} 3, {$4$} 4, {$5$} 5}
\normalsize
\tcaption{The graph of $y = f(x)$.}
\penwd{1.25pt}
\point[4pt]{(-5, -5), (-4, 0), (-3, 4), (-2,2), (-1,0), (0,-1), (1,0), (2,3), (3,1)}
\polyline{(-5,-5), (-4,0), (-3,4), (-2,2), (-1,0)}
\function{-1, 1, 0.1}{(x**2)-1}
\polyline{(1,0), (2,3), (3,1)}
\end{mfpic}



\begin{mfpic}[15]{-5}{5}{-6}{6}

\axes
\tlabel[cc](5,-0.5){\scriptsize $t$}
\tlabel[cc](0.5,6){\scriptsize $s$}
\xmarks{-3,-2,-1,1,2,3,4}
\ymarks{-5,-4,-3,-2,-1,1,2,3,4,5}
\tlpointsep{5pt}
\scriptsize
\axislabels {x}{{$-4 \hspace{7pt}$} -4, {$-3 \hspace{7pt}$} -3,{$-2 \hspace{7pt}$} -2, {$-1 \hspace{7pt}$} -1, {$1$} 1, {$2$} 2, {$3$} 3, {$4$} 4}
\axislabels {y}{{$-5$} -5,{$-4$} -4,{$-3$} -3,{$-2$} -2,{$-1$} -1, {$1$} 1, {$2$} 2, {$3$} 3, {$4$} 4, {$5$} 5}
\normalsize
\tcaption{The graph of $s = g(t)$.}
\penwd{1.25pt}
\function{-4, 4, 0.1}{5*sin(x*3.14159/4)}
\point[4pt]{ (-2, -5), (0, 0), (4,0), (2,3), (-4,0)}
\pointfillfalse
\point[4pt]{(2,5)}
\end{mfpic}

\end{multicols}

\begin{multicols}{4}

\begin{enumerate}

\setcounter{enumi}{\value{HW}}

\item \label{functionvaluesfromgraphfirst} $f(-2)$

\item $g(-2)$

\item $f(2)$

\item  $g(2)$

\setcounter{HW}{\value{enumi}}

\end{enumerate}

\end{multicols}

\begin{multicols}{4}

\begin{enumerate}

\setcounter{enumi}{\value{HW}}

\item $f(0)$

\item $g(0)$

\item  Solve $f(x) = 0$.

\item  Solve $g(t) = 0$. 

\setcounter{HW}{\value{enumi}}

\end{enumerate}

\end{multicols}

\begin{multicols}{2}

\begin{enumerate}

\setcounter{enumi}{\value{HW}}

\item  State the domain and range of $f$.

\item  State the domain and range of $g$ . \label{functionvaluesfromgraphlast}

\setcounter{HW}{\value{enumi}}

\end{enumerate}

\end{multicols}

\newpage

In Exercises \ref{sketchgraphfirst} - \ref{sketchgraphlast}, graph each function by making a table, plotting points, and using a graphing utility (if needed.)  Use the independent variable as the horizontal axis label and the default `$y$' label for the vertical axis label.  State the domain and range of each function. 
\begin{multicols}{3}
\begin{enumerate}
\setcounter{enumi}{\value{HW}}
\item $f(x) = 2-x$ \label{sketchgraphfirst}
\item $g(t) = \dfrac{t - 2}{3}$
\item $h(s) = s^2 + 1$

\setcounter{HW}{\value{enumi}}
\end{enumerate}
\end{multicols}

\begin{multicols}{3}
\begin{enumerate}
\setcounter{enumi}{\value{HW}}

\item $f(x) = 4-x^2$
\item $g(t) = 2$
\item $h(s) = s^3$

\setcounter{HW}{\value{enumi}}
\end{enumerate}
\end{multicols}

\begin{multicols}{3}
\begin{enumerate}
\setcounter{enumi}{\value{HW}}

\item $f(x) = x(x-1)(x+2)$
\item $g(t) = \sqrt{t-2}$
\item $h(s) = \sqrt{5 - s}$

\setcounter{HW}{\value{enumi}}
\end{enumerate}
\end{multicols}

\begin{multicols}{3}
\begin{enumerate}
\setcounter{enumi}{\value{HW}}

\item $f(x) = 3-2\sqrt{x+2}$
\item $g(t) = \sqrt[3]{t}$
\item $h(s) = \dfrac{1}{s^{2} + 1}$ \label{sketchgraphlast}

\setcounter{HW}{\value{enumi}}
\end{enumerate}
\end{multicols}



\begin{enumerate}
\setcounter{enumi}{\value{HW}}


\item Consider the function $f$ described below:

\begin{center}

\begin{mfpic}[19]{-5}{5}{-5}{6}
\tlabel[cc](-3,5){$-1$}
\tlabel[cc](-3,3){0}
\tlabel[cc](-3,1){1}
\tlabel[cc](-3,-1){2}
\tlabel[cc](0,6){$f$}
\tlabel[cc](3.5,5){$-3$}
\tlabel[cc](3.5,3){0}
\tlabel[cc](3.5,1){4}
\arrow[l 5pt] \polyline{(-2.5, 5), (2.5, 3)}
\arrow[l 5pt] \polyline{(-2.5, 3), (2.5, 5)}
\arrow[l 5pt] \polyline{(-2.5, 1), (2.5, 3)}
\arrow[l 5pt] \polyline{(-2.5, -1), (2.5, 1)}
\end{mfpic}

\end{center}

\begin{enumerate}

\item  State  the domain and range of $f$.

\item Find $f(0)$ and solve $f(x) = 0$.

\item  Write $f$ as a set of ordered pairs.

\item  Graph $f$.

\end{enumerate}


\item  Let $g = \{ (-1,4), (0,2), (2, 3), (3,4)  \}$

\begin{enumerate}

\item  State the domain and range of $g$.

\item  Create a mapping diagram for $g$.

\item  Find $g(0)$ and solve $g(x) = 0$.

\item  Graph $g$.


\end{enumerate}

\item  Let $F = \{ (t, t^2) \, | \, \text{$t$ is a real number} \}$.  Find $F(4)$ and solve $F(x) = 4$.

\textbf{HINT:}  Elements of $F$ are of the form $(x, F(x))$.

\item  Let $G = \{ (2t, t+5) \, | \, \text{$t$ is a real number} \}$.  Find $G(4)$ and solve $G(x) = 4$.

\textbf{HINT:}  Elements of $G$ are of the form $(x, G(x))$.

\setcounter{HW}{\value{enumi}}

\end{enumerate}



\begin{enumerate}
\setcounter{enumi}{\value{HW}}

\item  The area enclosed by a square, in square inches,  is a function of the length of one of its sides $\ell$, when measured in inches.  This function is represented by the formula $A(\ell) = \ell^2$ for $\ell > 0$.  Find $A(3)$ and solve $A(\ell) = 36$.  Interpret your answers to each.  Why is $\ell$ restricted to $\ell > 0$?

\item  The area enclosed by a circle, in square meters, is a function of its radius $r$, when measured in meters.  This function is represented by the formula $A(r) = \pi r^2$ for $r > 0$.  Find $A(2)$ and solve $A(r) = 16\pi$.  Interpret your answers to each.  Why is $r$ restricted to $r > 0$?

\item  The volume  enclosed by a cube, in cubic centimeters, is a function of the length of one of its sides $s$, when measured in centimeters.   This function is represented by the formula $V(s) = s^3$ for $s > 0$.  Find $V(5)$ and solve $V(s) = 27$.  Interpret your answers to each.  Why is $s$ restricted to $s > 0$?

\item  The volume enclosed by a sphere, in cubic feet, is a function of the radius of the sphere $r$, when measured in feet. This function is represented by the formula  $V(r) =\frac{4\pi}{3} r^{3}$ for $r > 0$.  Find $V(3)$ and solve $V(r) = \frac{32\pi}{3}$.  Interpret your answers to each.  Why is $r$ restricted to $r > 0$?

\item  The height of an object dropped from the roof of an eight story building is modeled by the function:  $h(t) = -16t^2 + 64$, $0 \leq t \leq 2$. Here,  $h(t)$ is the height of the object off the ground, in feet, $t$ seconds after the object is dropped.  Find $h(0)$ and solve $h(t) = 0$.  Interpret your answers to each.  Why is $t$ restricted to $0 \leq t \leq 2$?

\item  The temperature in degrees Fahrenheit $t$ hours after 6 AM is given by $T(t) = -\frac{1}{2} t^2 + 8t+3$ for $0 \leq t \leq 12$. Find and interpret $T(0)$, $T(6)$ and $T(12)$.  

\item The function $C(x) = x^2-10x+27$  models the cost, in \textit{hundreds} of dollars, to produce $x$ \textit{thousand} pens.  Find and interpret $C(0)$, $C(2)$ and $C(5)$.

\item Using data from the  \href{http://www.bts.gov/publications/national_transportation_statistics/html/table_04_23.html}{\underline{Bureau of Transportation Statistics}}, the average fuel economy in miles per gallon for passenger cars in the US can be modeled by  $E(t) = -0.0076t^2+0.45t + 16$, $0 \leq t \leq 28$, where $t$ is the number of years since $1980$. Use a calculator to find $E(0)$, $E(14)$ and $E(28)$.  Round your answers to two decimal places and interpret your answers to each.

\item  The perimeter of a square, in centimeters,  is four times the length of one if its sides, also measured in centimeters.  Represent the function $P$ which takes as its input the length of the side of a square in centimeters, $s$ and returns the perimeter of the square in inches, $P(s)$ using a formula.

\item  The circumference of a circle, in feet,  is $\pi$ times the diameter of the circle, also measured in feet.  Represent the function $C$ which takes as its input the length of the diameter of a circle in feet, $D$ and returns the circumference of a circle in inches, $C(D)$ using a formula.

\newpage

\item  Suppose $A(P)$ gives the amount of money in a retirement account (in dollars) after 30 years as a function of the amount of the monthly payment (in dollars), $P$.

\begin{enumerate}

\item What does $A(50)$ mean?  

\item  What is the significance of the solution to the equation $A(P) = 250000$? .

\item  Explain what each of the following expressions mean:  $A(P+50)$, $A(P)+50$, and $A(P) + A(50)$.  

\end{enumerate}

\item  Suppose $P(t)$ gives the chance of precipitation (in percent)  $t$ hours after 8 AM.

\begin{enumerate}

\item  Write an expression which gives the chance of precipitation at noon.

\item  Write an inequality which determines when the chance of precipitation is more than $50 \%$.

\end{enumerate}


\item Explain why the graph in  Exercise \ref{graphfunctionfirstvw}  suggests that not only is $v$ as a function of $w$ but also $w$ is a function of $v$.  Suppose $w = f(v)$ and $v = g(w)$.  That is, $f$ is the name of the function which takes $v$ values as inputs and returns $w$ values as outputs and $g$ is the name of the function which does vice-versa.   Find the domain and range of  $g$ and compare these to the domain and range of $f$.  

\item  Sketch the graph of a function with domain $(-\infty, 3) \cup [4,5)$ with range $\{ 2 \} \cup (5, \infty)$.

\end{enumerate}



\newpage


\subsection{Answers}

\begin{enumerate}

\item  The mapping $M$ is not a function since  `Tennant' is matched with both `Eleven' and `Twelve.'

\item  The mapping $C$ is a function since each input is matched with only one output.  The domain of $C$ is $\{ \text{Hartnell}, \text{Cushing}, \text{Hurndall}, \text{Troughton} \}$ and the range is $\{\text{One}, \text{Two} \}$.  We can represent $C$ as the following set of ordered pairs:  $\{ (\text{Hartnell}, \text{One}), (\text{Cushing}, \text{One}),  (\text{Hurndall}, \text{One}), (\text{Troughton}, \text{Two}) \}$

\setcounter{HW}{\value{enumi}}
\end{enumerate}

\begin{enumerate}

\setcounter{enumi}{\value{HW}}

\item  In this case, $y$ is a function of $x$ since each $x$ is matched with only one $y$.  

The domain is $\{ -3, -2, -1,0,1,2,3 \}$ and the range is $\{ 0,1,2,3 \}$.  

As ordered pairs, this function is $\{ (-3,3), (-2,2), (-1,1), (0,0), (1,1), (2,2), (3,3) \}$

\item  In this case, $y$ is not a function of $x$ since there are $x$ values matched with more than one $y$ value.  For instance, $1$ is matched both to $1$ and $-1$.

\setcounter{HW}{\value{enumi}}
\end{enumerate}


\begin{enumerate}

\setcounter{enumi}{\value{HW}}

\item    The mapping is a function since given any word, there is only one answer to `how many letters are in the word?'  For the range, we would need to know what the length of the longest word is and whether or not we could find words of all the lengths between $1$ (the length of the word `a') and it.  See \href{https://en.wikipedia.org/wiki/Longest_word_in_English}{\underline{here}}.

\item  Since Grover Cleveland was both the 22nd and 24th POTUS, neither mapping described in this exercise is a function.

\item  The outdoor temperature could never be the same for more than two different times - so, for example, it could always be getting warmer or it could always be getting colder.

\setcounter{HW}{\value{enumi}}

\end{enumerate}

\begin{multicols}{2}
\begin{enumerate}
\setcounter{enumi}{\value{HW}}

\item $f(2) = \frac{7}{4}$, $f(x) = \frac{2x+3}{4}$

\item $f(2) = \frac{5}{2}$, $f(x) = \frac{2(x+3)}{4} = \frac{x+3}{2}$  

\setcounter{HW}{\value{enumi}}
\end{enumerate}
\end{multicols}

\begin{multicols}{2}
\begin{enumerate}
\setcounter{enumi}{\value{HW}}

\item $f(2) = 7$, $f(x) = 2\left(\frac{x}{4} + 3\right) = \frac{1}{2} x + 6$   

\item $f(2) = \sqrt{7}$, $f(x) = \sqrt{2x+3}$ 

\setcounter{HW}{\value{enumi}}
\end{enumerate}
\end{multicols}

\begin{multicols}{2}
\begin{enumerate}
\setcounter{enumi}{\value{HW}}

\item $f(2) = \sqrt{10}$, $f(x) = \sqrt{2(x+3)} = \sqrt{2x+6}$

\item $f(2) = 2 \sqrt{5}$, $f(x) = 2\sqrt{x+3}$ 

\setcounter{HW}{\value{enumi}}
\end{enumerate}
\end{multicols}

\begin{enumerate}
\setcounter{enumi}{\value{HW}}

\item For $f(x) = 2x+1$ 

\begin{multicols}{3}
\begin{itemize}
\item $f(3) = 7$
\item $f(-1) = -1$
\item $f\left(\frac{3}{2} \right) = 4$
\end{itemize}
\end{multicols}

\begin{multicols}{3}
\begin{itemize}
\item  $f(4x) = 8x+1$
\item $4f(x) = 8x+4$
\item $f(-x) = -2x+1$
\end{itemize}
\end{multicols}

\begin{multicols}{3}
\begin{itemize}
\item  $f(x-4) = 2x-7$
\item $f(x) - 4 = 2x-3$
\item  $f\left(x^2\right) = 2x^2+1$
\end{itemize}
\end{multicols}

\item For $f(x) = 3-4x$ 

\begin{multicols}{3}
\begin{itemize}
\item $f(3) = -9$
\item $f(-1) = 7$
\item $f\left(\frac{3}{2} \right) = -3$
\end{itemize}
\end{multicols}

\begin{multicols}{3}
\begin{itemize}
\item  $f(4x) = 3-16x$
\item $4f(x) = 12-16x$
\item $f(-x) = 4x+3$
\end{itemize}
\end{multicols}

\begin{multicols}{3}
\begin{itemize}
\item  $f(x-4) = 19-4x$
\item $f(x) - 4 = -4x-1$
\item  $f\left(x^2\right) = 3-4x^2$
\end{itemize}
\end{multicols}



\item For $f(x) = 2 - x^2$ 

\begin{multicols}{3}
\begin{itemize}
\item $f(3) = -7$
\item $f(-1) = 1$
\item $f\left(\frac{3}{2} \right) = -\frac{1}{4}$
\end{itemize}
\end{multicols}

\begin{multicols}{3}
\begin{itemize}
\item  $f(4x) = 2-16x^2$
\item $4f(x) = 8-4x^2$
\item $f(-x) = 2-x^2$
\end{itemize}
\end{multicols}

\begin{multicols}{3}
\begin{itemize}
\item  $f(x-4) = -x^2+8x-14$
\item $f(x) - 4 = -x^{2} - 2$
\item  $f\left(x^2\right) = 2-x^4$
\end{itemize}
\end{multicols}

\item For $f(x) = x^2 - 3x + 2$ 

\begin{multicols}{3}
\begin{itemize}
\item $f(3) = 2$
\item $f(-1) = 6$
\item $f\left(\frac{3}{2} \right) = -\frac{1}{4}$
\end{itemize}
\end{multicols}

\begin{multicols}{3}
\begin{itemize}
\item  $f(4x) = 16x^2-12x+2$
\item $4f(x) = 4x^2-12x+8$
\item $f(-x) = x^2+3x+2$
\end{itemize}
\end{multicols}

\begin{multicols}{3}
\begin{itemize}
\item  $f(x-4) = x^2-11x+30$
\item $f(x) - 4 = x^2-3x-2$
\item  $f\left(x^2\right) = x^4-3x^2+2$
\end{itemize}
\end{multicols}



\item For $f(x) = 6$ 

\begin{multicols}{3}
\begin{itemize}
\item $f(3) = 6$
\item $f(-1) =6$
\item $f\left(\frac{3}{2} \right) = 6$
\end{itemize}
\end{multicols}

\begin{multicols}{3}
\begin{itemize}
\item  $f(4x) = 6$
\item $4f(x) = 24$
\item $f(-x) = 6$
\end{itemize}
\end{multicols}

\begin{multicols}{3}
\begin{itemize}

\item  $f(x-4) = 6$ 

\item $f(x) - 4 = 2$
     
\item  $f\left(x^2\right) = 6$

\end{itemize}
\end{multicols}



\item For $f(x) = 0$ 

\begin{multicols}{3}
\begin{itemize}
\item $f(3) = 0$
\item $f(-1) =0$
\item $f\left(\frac{3}{2} \right) = 0$
\end{itemize}
\end{multicols}

\begin{multicols}{3}
\begin{itemize}
\item  $f(4x) = 0$
\item $4f(x) = 0$
\item $f(-x) = 0$
\end{itemize}
\end{multicols}

\begin{multicols}{3}
\begin{itemize}

\item  $f(x-4) = 0$ 

\item $f(x) - 4 = -4$
     
\item  $f\left(x^2\right) = 0$

\end{itemize}
\end{multicols}

\setcounter{HW}{\value{enumi}}
\end{enumerate}




\begin{enumerate}
\setcounter{enumi}{\value{HW}}

\item For $f(x) = 2x-5$

\begin{multicols}{3}
\begin{itemize}

\item  $f(2) = -1$
\item  $f(-2) = -9$
\item  $f(2a) = 4a-5$

\end{itemize}
\end{multicols}

\begin{multicols}{3}
\begin{itemize}

\item  $2 f(a) = 4a-10$
\item $f(a+2) = 2a-1$
\item $f(a) + f(2) = 2a-6$

\end{itemize}
\end{multicols}

\begin{multicols}{3}
\begin{itemize}

\item  $f \left( \frac{2}{a} \right) = \frac{4}{a} - 5$ \\
$\hphantom{f \left( \frac{2}{a} \right)} = \frac{4-5a}{a}$

\vfill

\columnbreak

\item $\frac{f(a)}{2} =\frac{2a-5}{2}$

\vfill

\columnbreak


\item  $f(a + h) = 2a + 2h - 5$

\end{itemize}
\end{multicols}

\item For $f(x) = 5-2x$

\begin{multicols}{3}
\begin{itemize}

\item  $f(2) = 1$
\item  $f(-2) = 9$
\item  $f(2a) = 5-4a$

\end{itemize}
\end{multicols}

\begin{multicols}{3}
\begin{itemize}

\item  $2 f(a) = 10-4a$
\item $f(a+2) = 1-2a$
\item $f(a) + f(2) = 6-2a$

\end{itemize}
\end{multicols}

\begin{multicols}{3}
\begin{itemize}

\item  $f \left( \frac{2}{a} \right) = 5 - \frac{4}{a}$ \\
$\hphantom{f \left( \frac{2}{a} \right)} = \frac{5a-4}{a}$

\vfill

\columnbreak

\item $\frac{f(a)}{2} = \frac{5-2a}{2}$

\vfill

\columnbreak


\item  $f(a + h) = 5-2a-2h$

\end{itemize}
\end{multicols}


\item For $f(x) = 2x^2-1$

\begin{multicols}{3}
\begin{itemize}

\item  $f(2) = 7$
\item  $f(-2) = 7$
\item  $f(2a) = 8a^2-1$

\end{itemize}
\end{multicols}

\begin{multicols}{3}
\begin{itemize}

\item  $2 f(a) = 4a^2-2$
\item $f(a+2) = 2a^2+8a+7$
\item $f(a) + f(2) = 2a^2+6$

\end{itemize}
\end{multicols}

\begin{multicols}{3}
\begin{itemize}

\item  $f \left( \frac{2}{a} \right) = \frac{8}{a^2} - 1$ \\
$\hphantom{f \left( \frac{2}{a} \right)} = \frac{8-a^2}{a^2}$

\vfill

\columnbreak

\item $\frac{f(a)}{2} =  \frac{2a^2-1}{2}$

\vfill

\columnbreak


\item  $f(a + h) = 2a^2+4ah+2h^2-1$

\end{itemize}
\end{multicols}



\item For $f(x) = 3x^2+3x-2$

\begin{multicols}{3}
\begin{itemize}

\item  $f(2) = 16$
\item  $f(-2) = 4$
\item  $f(2a) = 12a^2+6a-2$

\end{itemize}
\end{multicols}

\begin{multicols}{3}
\begin{itemize}

\item  $2 f(a) = 6a^2+6a-4$
\item $f(a+2) = 3a^2+15a+16$
\item \small $f(a) + f(2) = 3a^2+3a+14$ \normalsize

\end{itemize}
\end{multicols}

\begin{multicols}{3}
\begin{itemize}

\item  $f \left( \frac{2}{a} \right) = \frac{12}{a^2} + \frac{6}{a} - 2$ \\
$\hphantom{f \left( \frac{2}{a} \right)} = \frac{12+6a-2a^2}{a^2}$

\vfill

\columnbreak

\item $\frac{f(a)}{2} =  \frac{3a^2+3a-2}{2}$

\vfill

\columnbreak


\item  $f(a + h) = 3a^2 + 6ah + 3h^2+3a+3h-2$

\end{itemize}
\end{multicols}



\item For $f(x) = 117$

\begin{multicols}{3}
\begin{itemize}

\item  $f(2) = 117$
\item  $f(-2) = 117$
\item  $f(2a) = 117$

\end{itemize}
\end{multicols}

\begin{multicols}{3}
\begin{itemize}

\item  $2 f(a) = 234$
\item $f(a+2) = 117$
\item $f(a) + f(2) = 234$

\end{itemize}
\end{multicols}

\begin{multicols}{3}
\begin{itemize}

\item  $f \left( \frac{2}{a} \right) = 117$ 

\vfill

\columnbreak

\item $\frac{f(a)}{2} = \frac{117}{2}$

\vfill

\columnbreak


\item  $f(a + h) = 117$

\end{itemize}
\end{multicols}



\item For $f(x) = \frac{x}{2}$

\begin{multicols}{3}
\begin{itemize}

\item  $f(2) = 1$
\item  $f(-2) = -1$
\item  $f(2a) = a$

\end{itemize}
\end{multicols}

\begin{multicols}{3}
\begin{itemize}

\item  $2 f(a) = a$

\item $f(a+2) = \frac{a+2}{2}$

\vfill

\columnbreak

\item $f(a) + f(2) = \frac{a}{2}+ 1$ \\
      $\hphantom{f(a) + f(2)} = \frac{a+2}{2}$

\end{itemize}
\end{multicols}

\begin{multicols}{3}
\begin{itemize}

\item  $f \left( \frac{2}{a} \right) = \frac{1}{a}$

\vfill

\columnbreak

\item $\frac{f(a)}{2} =  \frac{a}{4}$

\vfill

\columnbreak


\item  $f(a + h) = \frac{a+h}{2}$

\end{itemize}
\end{multicols}




\setcounter{HW}{\value{enumi}}
\end{enumerate}

\begin{enumerate}
\setcounter{enumi}{\value{HW}}

\item For $f(x) = 2x-1$,  $f(0) = -1$ and $f(x) = 0$ when $x = \frac{1}{2}$

\item For $f(x) =  3 - \frac{2}{5} x$, $f(0) = 3$ and $f(x) = 0$ when $x = \frac{15}{2}$

\item For $f(x) =  2x^2-6$, $f(0) = -6$ and $f(x) = 0$ when $x = \pm \sqrt{3}$

\item For $f(x) =  x^2-x-12$, $f(0) = -12$ and $f(x) = 0$ when $x = -3$ or $x=4$


\setcounter{HW}{\value{enumi}}
\end{enumerate}

\begin{multicols}{3}
\begin{enumerate}
\setcounter{enumi}{\value{HW}}


\item Function
\item Function
\item Function

\setcounter{HW}{\value{enumi}}
\end{enumerate}
\end{multicols}

\begin{multicols}{3}
\begin{enumerate}
\setcounter{enumi}{\value{HW}}


\item Not a function
\item Function
\item Not a function

\setcounter{HW}{\value{enumi}}
\end{enumerate}
\end{multicols}

\begin{multicols}{3}
\begin{enumerate}
\setcounter{enumi}{\value{HW}}

\item  Not a function
\item  Function
\item  Not a function

\setcounter{HW}{\value{enumi}}
\end{enumerate}
\end{multicols}

\begin{multicols}{3}
\begin{enumerate}
\setcounter{enumi}{\value{HW}}

\item   Function
\item   Not a function
\item Function

\setcounter{HW}{\value{enumi}}
\end{enumerate}
\end{multicols}

\begin{multicols}{3}
\begin{enumerate}
\setcounter{enumi}{\value{HW}}

\item Function
\item  Function
\item Not a function

\setcounter{HW}{\value{enumi}}
\end{enumerate}
\end{multicols}


\begin{multicols}{2}
\begin{enumerate}
\setcounter{enumi}{\value{HW}}

\item Function \\ domain = \{$-3$, $-2$, $-1$, $0$, $1$, $2$ ,$3$\} \\ range = \{$0$, $1$, $4$, $9$\}

\vfill

\columnbreak

\item Not a function

\setcounter{HW}{\value{enumi}}
\end{enumerate}
\end{multicols}

\begin{multicols}{2}
\begin{enumerate}
\setcounter{enumi}{\value{HW}}

\item  Function \\ domain = $\left\{ -7, -3, 3, 4, 5, 6 \right\}$ \\ range = $\left\{ 0,4,5,6,9 \right\}$


\vfill

\columnbreak

\item  Function \\ domain =   $\left\{ 1, 4, 9, 16, 25, 36, \ldots \right\} \\ = \left\{ x \, | \, \text{$x$ is a perfect square} \right\}$ \\ range =  $\left\{ 2, 4, 6, 8, 10, 12, \ldots \right\} \\ = \left\{ y \, | \, \text{$y$ is a positive even integer} \right\}$

\setcounter{HW}{\value{enumi}}
\end{enumerate}
\end{multicols}

\begin{multicols}{2}
\begin{enumerate}
\setcounter{enumi}{\value{HW}}

\item  Not a function

\vfill

\columnbreak

\item Function \\ domain $= \{x \, | \, \text{$x$ is irrational} \}$ \\ range $= \{ 1\}$

\setcounter{HW}{\value{enumi}}
\end{enumerate}
\end{multicols}

\begin{multicols}{2}
\begin{enumerate}
\setcounter{enumi}{\value{HW}}

\item Function \\ domain  $= \{x \, | \, 1, 2, 4, 8, \ldots \}$ \\ $= \{x \, | \, \text{$x = 2^{n}$ for some whole number $n$} \}$ \\ range $= \{ 0, 1, 2, 3, \ldots \}$ \\ $= \{y \, | \, \text{$y$ is any whole number}\}$

\vfill

\columnbreak

\item Function \\ domain $= \{x \, | \, \text{$x$ is any integer} \}$ \\ range $= \{y \, | \, \text{ $y$ is the square of an integer}\}$

\setcounter{HW}{\value{enumi}}
\end{enumerate}
\end{multicols}

\begin{multicols}{2}
\begin{enumerate}
\setcounter{enumi}{\value{HW}}

\item Not a function

\vfill

\columnbreak

\item Function \\ domain  $= \{x \, | \, -2 \leq x < 4 \} = [-2, 4)$, \\ range = \{$3$\}

\setcounter{HW}{\value{enumi}}
\end{enumerate}
\end{multicols}

\begin{multicols}{2}
\begin{enumerate}
\setcounter{enumi}{\value{HW}}


\item Function \\ domain $= \{x \, | \,  \text{$x$ is a real number} \} = (-\infty, \infty)$ \\  range $= \{y \, | \,  y \geq 0 \} = [0,\infty)$

\vfill

\columnbreak

\item  Not a function

\setcounter{HW}{\value{enumi}}
\end{enumerate}
\end{multicols}

\begin{enumerate}
\setcounter{enumi}{\value{HW}}

 \item \textbf{Horizontal Line Test:} A graph on the $xy$-plane represents $x$ as a function of $y$ if and only if no \textbf{horizontal} line intersects the graph more than once.
 
 \setcounter{HW}{\value{enumi}}
\end{enumerate}






\begin{multicols}{2}
\begin{enumerate}
\setcounter{enumi}{\value{HW}}

\item Function \\ domain = \{$-4$, $-3$, $-2$, $-1$, $0$, $1$\} \\ range = \{$-1$, $0$, $1$, $2$, $3$, $4$\}

\vfill

\columnbreak

\item Not a function

\setcounter{HW}{\value{enumi}}
\end{enumerate}
\end{multicols}


\begin{multicols}{2}
\begin{enumerate}
\setcounter{enumi}{\value{HW}}

\item Function \\ domain = $(-\infty, \infty)$ \\ range = $[1, \infty)$
\vfill

\columnbreak

\item Not a function 

\setcounter{HW}{\value{enumi}}
\end{enumerate}
\end{multicols}

\begin{enumerate}
\setcounter{enumi}{\value{HW}}

\item  

\begin{itemize}

\item Number \ref{graphfunctionfirstxy} represents $x$ as a function of $y$.

domain  = \{$-1$, $0$, $1$, $2$, $3$, $4$\} and range = \{$-4$, $-3$, $-2$, $-1$, $0$, $1$\}

\item Number \ref{graphfunctionlastxy} represents $x$ as a function of $y$.

domain = $(-\infty, \infty)$  and range = $[1, \infty)$


\end{itemize}

\setcounter{HW}{\value{enumi}}
\end{enumerate}

\begin{multicols}{2}
\begin{enumerate}
\setcounter{enumi}{\value{HW}}

\item Function \\ domain = $[2, \infty)$ \\ range = $[0, \infty)$

\vfill

\columnbreak

\item Function \\ domain = $(-\infty, \infty)$ \\ range = $(0, 4]$

\setcounter{HW}{\value{enumi}}
\end{enumerate}
\end{multicols}


\begin{multicols}{2}
\begin{enumerate}
\setcounter{enumi}{\value{HW}}

\item Not a function


\vfill

\columnbreak

\item Function \\ domain = $[-5,-3) \cup(-3, 3)$ \\ range = $(-2, -1) \cup [0, 4)$

\setcounter{HW}{\value{enumi}}
\end{enumerate}
\end{multicols}

\begin{enumerate}
\setcounter{enumi}{\value{HW}}

\item Only number \ref{graphfunctionfirstvw} represents $v$ as a function of $w$;  domain = $[0, \infty)$ and range = $[2, \infty)$

\setcounter{HW}{\value{enumi}}
\end{enumerate}

\begin{multicols}{2}
\begin{enumerate}
\setcounter{enumi}{\value{HW}}

\item  Function \\  domain =  $[-2, \infty)$ \\ range = $[-3, \infty)$

\vfill

\columnbreak

\item Not a function

\setcounter{HW}{\value{enumi}}
\end{enumerate}
\end{multicols}


\begin{multicols}{2}
\begin{enumerate}
\setcounter{enumi}{\value{HW}}

\item Function \\  domain =  $(-5, 4)$ \\ range = $(-4, 4)$

\vfill

\columnbreak

\item  Function \\ domain = $[0,3) \cup (3,6]$ \\ range = $(-4,-1] \cup [0,4]$

\setcounter{HW}{\value{enumi}}
\end{enumerate}
\end{multicols}

\begin{enumerate}
\setcounter{enumi}{\value{HW}}

\item None of numbers \ref{graphfunctionfirsttT} - \ref{graphfunctionlasttT}  represent $t$ as a function of $T$.

\setcounter{HW}{\value{enumi}}
\end{enumerate}



\begin{multicols}{2}
\begin{enumerate}
\setcounter{enumi}{\value{HW}}

\item  Function \\ domain = $(-\infty, \infty)$ \\ range = $(-\infty, 4]$

\vfill

\columnbreak

\item  Function \\ domain = $(-\infty, \infty)$ \\ range = $(-\infty, 4]$

\setcounter{HW}{\value{enumi}}
\end{enumerate}
\end{multicols}


\begin{multicols}{2}
\begin{enumerate}
\setcounter{enumi}{\value{HW}}

\item  Function \\  domain =  $[-2, \infty)$  \\ range =   $(-\infty, 3]$

\vfill

\columnbreak

\item  Function \\ domain = $(-\infty, \infty)$ \\ range = $(-\infty, \infty)$

\setcounter{HW}{\value{enumi}}
\end{enumerate}
\end{multicols}

\begin{enumerate}
\setcounter{enumi}{\value{HW}}

\item Only number \ref{graphfunctionfirstHs3} represents $s$ as a function of $H$;  domain =  $(-\infty, 3]$  and range =   $[-2, \infty)$


\setcounter{HW}{\value{enumi}}
\end{enumerate}

\begin{multicols}{2}
\begin{enumerate}
\setcounter{enumi}{\value{HW}}

\item  Function \\  domain =  $(-\infty, 0] \cup (1, \infty)$ \\ range =  $(-\infty, 1] \cup \{ 2\}$

\vfill

\columnbreak

\item  Function \\  domain =  $[-3,3]$ \\ range =  $[-2,2]$

\setcounter{HW}{\value{enumi}}
\end{enumerate}
\end{multicols}

\begin{multicols}{2}
\begin{enumerate}
\setcounter{enumi}{\value{HW}}

\item   Not a function

\vfill

\columnbreak

\item  Function \\ domain = $(-\infty, \infty)$ \\ range = $\{2\}$

\setcounter{HW}{\value{enumi}}
\end{enumerate}
\end{multicols}

\begin{enumerate}
\setcounter{enumi}{\value{HW}}

\item  Only number \ref{graphfunctionfirstut3} represents $t$ as a function of $u$;  domain = $(-\infty, \infty)$ and range=$\{2 \}$.

\setcounter{HW}{\value{enumi}}
\end{enumerate}


\begin{multicols}{4}

\begin{enumerate}

\setcounter{enumi}{\value{HW}}

\item  $f(-2) = 2$

\item $g(-2) = -5$

\item $f(2) = 3$

\item  $g(2) = 3$


\setcounter{HW}{\value{enumi}}

\end{enumerate}

\end{multicols}

\begin{multicols}{4}

\begin{enumerate}

\setcounter{enumi}{\value{HW}}

\item $f(0) = -1$

\item $g(0) = 0$

\item  $x  = -4, -1, 1$

\item  $t = -4, 0, 4$

\setcounter{HW}{\value{enumi}}

\end{enumerate}

\end{multicols}

\begin{multicols}{2}

\begin{enumerate}

\setcounter{enumi}{\value{HW}}

\item  Domain:  $[-5,3]$,  Range:  $[-5,4]$.

\item  Domain:  $[-4,4]$,  Range:  $[-5,5)$.

\setcounter{HW}{\value{enumi}}

\end{enumerate}

\end{multicols}



\begin{enumerate}

\setcounter{enumi}{\value{HW}}

\item \begin{multicols}{2} \raggedcolumns 

$f(x) =2-x$

Domain: $(-\infty, \infty)$ 

Range:  $(-\infty, \infty)$


\begin{mfpic}[15]{-3}{4}{-2}{4}

\axes
\tlabel[cc](4,-0.5){\scriptsize $x$}
\tlabel[cc](0.5,4){\scriptsize $y$}
\xmarks{-2,-1,1,2,3}
\ymarks{-1,1,2,3}
\tlpointsep{4pt}
\tiny 
\axislabels {x}{{$-2 \hspace{6pt}$} -2,{$-1 \hspace{6pt}$} -1, {$1$} 1, {$2$} 2, {$3$} 3}
\axislabels {y}{{$-1$} -1, {$1$} 1, {$2$} 2, {$3$} 3}
\normalsize
\penwd{1.25pt}
\arrow \reverse \arrow \function{-1.75,3.75, 0.1}{2-x}
\point[4pt]{(-1,3), (0, 2), (1,1), (2,0), (3,-1)}
\end{mfpic}

\end{multicols}

\item \begin{multicols}{2} \raggedcolumns 

$g(t) = \dfrac{t - 2}{3}$

Domain: $(-\infty, \infty)$ 

Range: $(-\infty, \infty)$  

\vfill

\columnbreak

\begin{mfpic}[15]{-2}{5}{-2}{2}
\axes
\tlabel[cc](5,-0.5){\scriptsize $t$}
\tlabel[cc](0.5,2){\scriptsize $y$}
\xmarks{-1,1,2,3,4}
\ymarks{-1,1}
\tlpointsep{4pt}
\tiny 
\axislabels {x}{{$-1 \hspace{6pt}$} -1, {$1$} 1, {$2$} 2, {$3$} 3, {$4$} 4}
\axislabels {y}{{$-1$} -1, {$1$} 1}
\normalsize
\penwd{1.25pt}
\arrow \reverse \arrow \function{-2,5, 0.1}{(x - 2)/3}
\point[4pt]{(-1,-1), (0, -0.6667), (1,-0.3333), (2,0), (3, 0.3333)}
\end{mfpic}

\end{multicols}

\item \begin{multicols}{2} \raggedcolumns 

$h(s) = s^2+1$

Domain: $(-\infty, \infty)$ 

Range:  $[1, \infty)$

\vfill

\columnbreak


\begin{mfpic}[15]{-3}{3}{-1}{6}
\axes
\tlabel[cc](3,-0.5){\scriptsize $s$}
\tlabel[cc](0.5,5.75){\scriptsize $y$}
\xmarks{-2,-1,1,2}
\ymarks{1,2,3,4,5}
\tlpointsep{4pt}
\axislabels {x}{{\tiny $-2 \hspace{8pt}$} -2, {\tiny $-1 \hspace{8pt}$} -1, {\tiny $1$} 1, {\tiny $2$} 2}
\axislabels {y}{{\tiny $1$} 1, {\tiny $2$} 2, {\tiny $3$} 3, {\tiny $4$} 4, {\tiny $5$} 5}
\penwd{1.25pt}
\arrow \reverse \arrow \function{-2.1, 2.1, 0.1}{x**2+1}
\point[4pt]{(-1,2), (0,1), (1,2), (2,5), (-2,5)}
\end{mfpic}

\end{multicols}

\item \begin{multicols}{2} \raggedcolumns 

$f(x) = 4-x^2$

Domain: $(-\infty, \infty)$ 

Range:  $(-\infty, 4]$

\vfill

\columnbreak


\begin{mfpic}[15]{-3}{3}{-1}{5}
\axes
\tlabel[cc](3,-0.5){\scriptsize $x$}
\tlabel[cc](0.5,5){\scriptsize $y$}
\xmarks{-2,-1,1,2}
\ymarks{1,2,3,4}
\tlpointsep{4pt}
\axislabels {x}{{\tiny $-2 \hspace{6pt}$} -2, {\tiny $-1 \hspace{6pt}$} -1, {\tiny $1$} 1, {\tiny $2$} 2}
\axislabels {y}{{\tiny $1$} 1, {\tiny $2$} 2, {\tiny $3$} 3, {\tiny $4$} 4}
\penwd{1.25pt}
\arrow \reverse \arrow \function{-2.25,2.25,0.1}{4-(x**2)}
\point[4pt]{(-1,3), (0,4), (1,3), (-2,0), (2,0)}
\end{mfpic} 

\end{multicols}

\item \begin{multicols}{2} \raggedcolumns 

$g(t) = 2$

Domain: $(-\infty, \infty)$ 

Range:  $\{2\}$

\vfill

\columnbreak


\begin{mfpic}[15]{-3}{3}{-1}{4}
\axes
\tlabel[cc](3,-0.5){\scriptsize $t$}
\tlabel[cc](0.5,4){\scriptsize $y$}
\xmarks{-2,-1,1,2}
\ymarks{1,2,3}
\tlpointsep{4pt}
\axislabels {x}{{\tiny $-2 \hspace{6pt}$} -2, {\tiny $-1 \hspace{6pt}$} -1, {\tiny $1$} 1, {\tiny $2$} 2}
\axislabels {y}{{\tiny $1$} 1, {\tiny $2$} 2, {\tiny $3$} 3}
\penwd{1.25pt}
\arrow \reverse \arrow \function{-3,3,0.1}{2}
\point[4pt]{(-2,2), (-1,2), (0,2), (1,2), (2,2)}
\end{mfpic} 

\end{multicols}


\item \begin{multicols}{2} \raggedcolumns 

$h(s) = s^3$ 

Domain: $(-\infty, \infty)$ 

Range:  $(-\infty, \infty)$


\begin{mfpic}[10]{-3}{3}{-9}{9}

\axes
\tlabel[cc](3,-0.5){\scriptsize $s$}
\tlabel[cc](0.5,9){\scriptsize $y$}
\ymarks{-8,-7,-6,-5,-4,-3,-2,-1,1,2,3,4,5,6,7,8}
\xmarks{-2,-1,1,2}
\tlpointsep{4pt}
\tiny 
\axislabels {x}{{$-2 \hspace{6pt}$} -2, {$-1 \hspace{6pt}$} -1, {$1$} 1, {$2$} 2}
\axislabels {y}{{$-8$} -8,{$-7$} -7,{$-6$} -6,{$-5$} -5,{$-4$} -4,{$-3$} -3,{$-2$} -2, {$-1$} -1, {$1$} 1, {$2$} 2, {$3$} 3, {$4$} 4, {$5$} 5, {$6$} 6, {$7$} 7, {$8$} 8}
\normalsize
\penwd{1.25pt}
\arrow \reverse \arrow \parafcn{-2.1,2.1,0.1}{(t,t**3)}
\point[4pt]{(0,0), (-1, -1), (1, 1), (-2, -8), (2, 8)}
\end{mfpic}

\end{multicols}

\item \begin{multicols}{2} \raggedcolumns 

$f(x) = x(x-1)(x+2)$

Domain: $(-\infty, \infty)$ 

Range: $(-\infty, \infty)$


\vfill

\columnbreak


\begin{mfpic}[15]{-3}{3}{-1}{5}
\axes
\tlabel[cc](3,-0.5){\scriptsize $x$}
\tlabel[cc](0.5,5){\scriptsize $y$}
\xmarks{-2,-1,1,2}
\ymarks{1,2,3,4}
\tlpointsep{4pt}
\axislabels {x}{{\tiny $-2 \hspace{6pt}$} -2, {\tiny $-1 \hspace{6pt}$} -1, {\tiny $1$} 1, {\tiny $2$} 2}
\axislabels {y}{{\tiny $1$} 1, {\tiny $2$} 2, {\tiny $3$} 3, {\tiny $4$} 4}
\penwd{1.25pt}
\arrow \reverse \arrow \function{-2.25, 1.75, 0.1}{x*(x-1)*(x+2)}
\point[4pt]{(-2,0), (-1,2), (0,0), (1,0)}
\end{mfpic} 
\end{multicols}


\item \begin{multicols}{2} \raggedcolumns 

$g(t) = \sqrt{t-2}$

Domain: $[2, \infty)$ 

Domain: $[0, \infty)$ 


\vfill

\columnbreak


\begin{mfpic}[15]{-1}{10}{-1}{4}
\axes
\tlabel[cc](10,-0.5){\scriptsize $t$}
\tlabel[cc](0.5,3.75){\scriptsize $y$}
\xmarks{1,2,3,4,5,6,7,8,9}
\ymarks{1,2,3}
\tlpointsep{4pt}
\axislabels {x}{{\tiny $1$} 1, {\tiny $2$} 2, {\tiny $3$} 3, {\tiny $4$} 4, {\tiny $5$} 5, {\tiny $6$} 6, {\tiny $7$} 7, {\tiny $8$} 8, {\tiny $9$} 9}
\axislabels {y}{{\tiny $1$} 1, {\tiny $2$} 2, {\tiny $3$} 3}
\penwd{1.25pt}
\arrow \function{2, 10, 0.1}{sqrt(x - 2)}
\point[4pt]{(2,0), (3,1), (6,2)}
\end{mfpic}

\end{multicols}

\item \begin{multicols}{2} \raggedcolumns 

$h(s) = \sqrt{5 - s}$ 

Domain: $(-\infty, 5]$ 

Range:  $[0, \infty)$

\begin{mfpic}[15]{-5}{6}{-1}{4}

\axes
\tlabel[cc](6,-0.5){\scriptsize $s$}
\tlabel[cc](0.5,3.75){\scriptsize $y$}
\xmarks{-4,-3,-2,-1,1,2,3,4,5}
\ymarks{1,2,3}
\tlpointsep{4pt}
\tiny 
\axislabels {x}{{$-4 \hspace{6pt}$} -4, {$-3 \hspace{6pt}$} -3, {$-2 \hspace{6pt}$} -2, {$-1 \hspace{6pt}$} -1, {$1$} 1, {$2$} 2, {$3$} 3, {$4$} 4, {$5$} 5}
\axislabels {y}{{$1$} 1, {$2$} 2, {$3$} 3}
\normalsize
\penwd{1.25pt}
\arrow \reverse \function{-4.5, 5, 0.1}{sqrt(5 - x)}
\point[4pt]{(5,0), (0, 2.2360679), (1, 2), (-4, 3)}
\end{mfpic}

\end{multicols}



\item \begin{multicols}{2} \raggedcolumns 

$f(x) = 3-2\sqrt{x+2}$ 

Domain: $[-2,\infty)$ 

Range: $(-\infty, 3]$ 


\vfill

\columnbreak

\begin{mfpic}[15]{-3}{3}{-1.5}{5}
\axes
\tlabel[cc](3,-0.5){\scriptsize $x$}
\tlabel[cc](0.5,5){\scriptsize $y$}
\xmarks{-2,-1,1,2}
\ymarks{1,2,3,4}
\tlpointsep{4pt}
\axislabels {x}{{\tiny $-2 \hspace{6pt}$} -2, {\tiny $-1 \hspace{6pt}$} -1, {\tiny $1$} 1, {\tiny $2$} 2}
\axislabels {y}{{\tiny $1$} 1, {\tiny $2$} 2, {\tiny $3$} 3, {\tiny $4$} 4}
\penwd{1.25pt}
\arrow \function{-2, 3, 0.1}{3-2*sqrt(x+2)}
\point[4pt]{(-2,3), (-1,1), (2,-1)} 
\end{mfpic} 

\end{multicols}


\item \begin{multicols}{2} \raggedcolumns 

$g(t) = \sqrt[3]{t}$ 

Domain: $(-\infty, \infty)$ 

Range:  $(-\infty, \infty)$

\columnbreak

\begin{mfpic}[10]{-9}{9}{-3}{3}
\point[4pt]{(0,0), (-1, -1), (1, 1), (-8, -2), (8, 2)}
\axes
\tlabel[cc](9,-0.5){\scriptsize $t$}
\tlabel[cc](0.5,2.75){\scriptsize $y$}
\xmarks{-8,-7,-6,-5,-4,-3,-2,-1,1,2,3,4,5,6,7,8}
\ymarks{-2,-1,1,2}
\tlpointsep{4pt}
\tiny 
\axislabels {x}{{$-8 \hspace{6pt}$} -8, {$-7 \hspace{6pt}$} -7, {$-6 \hspace{6pt}$} -6, {$-5 \hspace{6pt}$} -5, {$-4 \hspace{6pt}$} -4, {$-3 \hspace{6pt}$} -3, {$-2 \hspace{6pt}$} -2, {$-1 \hspace{6pt}$} -1, {$1$} 1, {$2$} 2, {$3$} 3, {$4$} 4, {$5$} 5, {$6$} 6, {$7$} 7, {$8$} 8}
\axislabels {y}{{$-2$} -2, {$-1$} -1, {$1$} 1, {$2$} 2}
\normalsize
\penwd{1.25pt}
\arrow \reverse \arrow \parafcn{-2.1,2.1,0.1}{(t**3,t)}
\point[4pt]{(0,0), (-1, -1), (1, 1), (-8, -2), (8, 2)}
\end{mfpic}

\end{multicols}

\item \begin{multicols}{2} \raggedcolumns 

$h(s) = \dfrac{1}{s^{2} + 1}$ 

Domain: $(-\infty, \infty)$ 

Range:  $(0, 1]$

\columnbreak

\begin{mfpic}[23]{-3}{3}{-1}{2}

\axes
\tlabel[cc](3,-0.5){\scriptsize $s$}
\tlabel[cc](0.5,1.75){\scriptsize $y$}
\xmarks{-2,-1,1,2}
\ymarks{1}
\tlpointsep{4pt}
\scriptsize
\axislabels {x}{{$-2 \hspace{7pt}$} -2, {$-1 \hspace{7pt}$} -1, {$1$} 1, {$2$} 2}
\axislabels {y}{{$1$} 1}
\normalsize
\penwd{1.25pt}
\arrow \reverse \arrow \function{-2.5, 2.5, 0.1}{1/(x**2 + 1)}
\point[4pt]{(0, 1), (1,0.5), (-1,0.5)}
\end{mfpic}

\end{multicols}

\setcounter{HW}{\value{enumi}}
\end{enumerate}


\begin{enumerate}
\setcounter{enumi}{\value{HW}}

\item \begin{multicols}{2} \raggedcolumns

\begin{enumerate}

\item domain $ = \{ -1, 0, 1, 2 \}$, range $ = \{ -3, 0, 4\}$

\item  $f(0) = -3$,  $f(x) = 0$ for $x = -1, 1$.

\item  $f = \{ (-1,0), (0, -3), (1,0), (2,4) \}$

\vfill 

\item  $~$

\begin{mfpic}[17]{-4}{4}{-5}{5}
\point[4pt]{(-1, 0), (0, -3), (1,0), (2, 4)}
\axes
\xmarks{-3, -2, -1, 0, 1, 2, 3}
\ymarks{-4 step 1 until 4}
\tlpointsep{4pt}
\axislabels {x}{{\tiny $-3 \hspace{8pt}$} -3, {\tiny $-2 \hspace{8pt}$} -2, {\tiny $-1 \hspace{8pt}$} -1,  {\tiny $1$} 1, {\tiny $2$} 2, {\tiny $3$} 3}
\axislabels {y}{{\tiny $-4$} -4, {\tiny $-3$} -3, {\tiny $-2$} -2, {\tiny $-1$} -1, {\tiny $1$} 1, {\tiny $2$} 2, {\tiny $3$} 3, {\tiny $4$} 4}
\end{mfpic}

\end{enumerate}

\end{multicols}

\item  \begin{multicols}{2} \raggedcolumns

\begin{enumerate}

\item  domain $= \{ -1, 0, 2, 3 \}$, range $=\{ 2, 3, 4 \}$

\item $~$

\begin{mfpic}[19]{-5}{5}{-5}{6}
\tlabel[cc](-3,5){-1}
\tlabel[cc](-3,3){0}
\tlabel[cc](-3,1){2}
\tlabel[cc](-3,-1){3}
\tlabel[cc](0,6){$g$}
\tlabel[cc](3.5,5){2}
\tlabel[cc](3.5,3){3}
\tlabel[cc](3.5,1){4}
\arrow[l 5pt] \polyline{(-2.5, 5), (2.5, 1)}
\arrow[l 5pt] \polyline{(-2.5, 3), (2.5, 5)}
\arrow[l 5pt] \polyline{(-2.5, 1), (2.5, 3)}
\arrow[l 5pt] \polyline{(-2.5, -1), (2.5, 1)}
\end{mfpic}

\vfill

\item  Find $g(0) = 2$ and $g(x) = 0$ has no solutions.



\item $~$

\begin{mfpic}[17]{-4}{4}{-1}{5}
\point[4pt]{(-1, 4), (0, 2), (2,3), (3, 4)}
\axes
\xmarks{-3, -2, -1, 0, 1, 2, 3}
\ymarks{1 step 1 until 4}
\tlpointsep{4pt}
\axislabels {x}{{\tiny $-3 \hspace{8pt}$} -3, {\tiny $-2 \hspace{8pt}$} -2, {\tiny $-1 \hspace{8pt}$} -1,  {\tiny $1$} 1, {\tiny $2$} 2, {\tiny $3$} 3}
\axislabels {y}{ {\tiny $1$} 1, {\tiny $2$} 2, {\tiny $3$} 3, {\tiny $4$} 4}
\end{mfpic}


\end{enumerate}

\end{multicols}

\item  $F(4) = 4^2 = 16$ (when $t = 4$), the solutions to $F(x) = 4$ are $x = \pm 2$ (when $t = \pm 2$). 

\item  $G(4) = 7$ (when $t = 2$), the solution to $G(t) = 4$ is $x = -2$ (when $t = -1$)


\setcounter{HW}{\value{enumi}}

\end{enumerate}


\begin{enumerate}
\setcounter{enumi}{\value{HW}}

\item  $A(3) = 9$, so the area enclosed by a square with a side of length $3$ inches is $9$ square inches.  The solutions to $A(\ell) = 36$ are $\ell = \pm 6$.  Since $\ell$ is restricted to  $\ell > 0$, we only keep $\ell  = 6$.  This means for the area enclosed by the square to be $36$ square inches, the length of the side needs to be $6$ inches.  Since $\ell$ represents a length, $\ell > 0$.



\item  $A(2) = 4\pi$, so the area enclosed by a circle with radius $2$ meters is $4\pi$ square meters.  The solutions to $A(r) = 16\pi$ are $r = \pm 4$.  Since $r$ is restricted to $r > 0$, we only keep $r = 4$.  This means for the area enclosed by the circle to be $16\pi$ square meters, the radius needs to be $4$ meters.  Since $r$ represents a radius (length), $r > 0$.

\item  $V(5) = 125$, so the volume enclosed by a cube with a side of length $5$ centimeters is $125$ cubic centimeters.  The solution to $V(s) = 27$ is $s = 3$.  This means for the volume enclosed by the cube to be $27$ cubic centimeters, the length of the side needs to $3$ centimeters.  Since $x$ represents a length, $x > 0$.

\item  $V(3) = 36\pi$, so the volume enclosed by a sphere with radius $3$ feet is $36\pi$ cubic feet.  The solution to $V(r) = \frac{32\pi}{3}$ is $r = 2$.  This means for the volume enclosed by the sphere to be $\frac{32\pi}{3}$ cubic feet, the radius needs to $2$ feet.  Since $r$ represents a radius (length), $r > 0$.


\item $h(0) = 64$, so at the moment the object is dropped off the building, the object is $64$ feet off of the ground.  The solutions to $h(t) = 0$ are $t = \pm 2$.  Since we restrict $0 \leq t \leq 2$, we only keep $t = 2$.  This means $2$ seconds after the object is dropped off the building, it is $0$ feet off the ground.  Said differently, the object hits the ground after $2$ seconds.  The restriction  $0 \leq t \leq 2$ restricts the time to be between the moment the object is released and the moment it hits the ground.


\item  $T(0) = 3$, so at 6 AM ($0$ hours after 6 AM), it is $3^{\circ}$ Fahrenheit.  $T(6) = 33$, so at noon ($6$ hours after 6 AM), the temperature is $33^{\circ}$ Fahrenheit.  $T(12) = 27$, so at 6 PM ($12$ hours after 6 AM), it is $27^{\circ}$ Fahrenheit.


\item $C(0) = 27$, so to make $0$ pens, it costs\footnote{This is called the `fixed' or `start-up' cost.  We'll revisit this concept in Example \ref{PortaBoyCost} in Section \ref{ConstantandLinearFunctions}.} $\$ 2700$.  $C(2) = 11$, so to make $2000$ pens, it costs $\$1100$.  $C(5) = 2$, so to make $5000$ pens, it costs $\$2000$.

\item $E(0) = 16.00$, so in 1980 ($0$ years after 1980), the average fuel economy of passenger cars in the US was $16.00$ miles per gallon.  $E(14) = 20.81$, so in 1994 ($14$ years after 1980), the average fuel economy of passenger cars in the US was $20.81$ miles per gallon.  $E(28) = 22.64$, so in 2008 ($28$ years after 1980), the average fuel economy of passenger cars in the US was $22.64$ miles per gallon.  


\item  $P(s) = 4s$, $s > 0$.

\item  $C(D) = \pi D$,  $D > 0$.

\item

\begin{enumerate}

\item The amount in the retirement account after 30 years if the monthly payment is $\$50$.

\item  The solution to $A(P) = 250000$ is what the monthly payment needs to be in order to have $\$250,\!000$ in the retirement account after 30 years.

\item  $A(P+50)$ is how much is in the retirement account in 30 years if $\$ 50$ is added to the monthly payment $P$.  $A(P)+50$ represents the amount of money in the retirement account after 30 years if $\$P$  is invested each month plus an additional $\$50$.  $A(P)+A(50)$ is the sum of money from two retirement accounts after 30 years: one with monthly payment $\$P$ and one with monthly payment $\$50$.

\end{enumerate}

\item  

\begin{enumerate}

\item  Since noon is $4$ hours after 8 AM, $P(4)$ gives the chance of precipitation at noon.

\item  We would need to solve $P(t) \geq 50 \%$ or $P(t) \geq 0.5$.

\end{enumerate}


\item The graph in question passes the horizontal line test meaning for each $w$ there is only one $v$.    The domain of $g$ is $[0, \infty)$ (which is the range of $f$) and the range of $g$ is $[2, \infty)$ which is the domain of $f$.  

\item  Answers vary.  

\end{enumerate}
