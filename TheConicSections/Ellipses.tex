\mfpicnumber{1}

\opengraphsfile{Ellipses}

\setcounter{footnote}{0}

\label{Ellipses}

In the definition of a circle, Definition \ref{circledefn}, we fixed a point called the \textbf{center} and considered all of the points which were a fixed distance $r$ from that one point.  For our next conic section, the ellipse, we fix two distinct points and a distance $d$ to use in our definition.  

\smallskip

\colorbox{ResultColor}{\bbm

 \begin{defn} \label{ellipsedefn} Given two distinct points $F_{\mbox{\tiny$1$}}$ and $F_{\mbox{\tiny$2$}}$ in the plane and a fixed distance $d$, an \index{ellipse ! definition of}  \textbf{ellipse} is the set of all points $(x, y)$ in the plane such that the sum of each of the distances from $F_{\mbox{\tiny$1$}}$ and $F_{\mbox{\tiny$2$}}$ to $(x, y)$ is $d$.  The points $F_{\mbox{\tiny$1$}}$ and $F_{\mbox{\tiny$2$}}$ are called the \index{ellipse ! foci} \index{focus (foci) ! of an ellipse} \textbf{foci}\footnote{the plural of `focus'} of the ellipse.

\end{defn} 

\ebm}

\smallskip

\begin{center}

\begin{mfpic}[20]{-6}{6}{-5}{5}

\dashed \polyline{(-3,0), (4,2.4)}
\dashed \polyline{(3,0), (4,2.4)}

\tlabel[cc](5, 2.4){$(x, y)$}
\tlabel[cc](-1,1.1){$d_{\mbox{\tiny$1$}}$}
\tlabel[cc](3,1.1){$d_{\mbox{\tiny$2$}}$}
\tlabel[cc](-3,-0.5){$F_{\mbox{\tiny$1$}}$}
\tlabel[cc](3,-0.5){$F_{\mbox{\tiny$2$}}$}
\plotsymbol[3pt]{Asterisk}{(-3,0)}
\plotsymbol[3pt]{Asterisk}{(3,0)}
\tcaption{$d_{\mbox{\tiny$1$}} + d_{\mbox{\tiny$2$}} = d$ for all $(x, y)$ on the ellipse}
\penwd{1.25pt}
\ellipse{(0,0),5,4}
\point[4pt]{(4,2.4)}
\end{mfpic}

\end{center}


We may imagine taking a length of string and anchoring it to two points on a piece of paper.  The curve traced out by taking a pencil and moving it so the string is always taut is an ellipse.  Each ellipse has an assortment of parameters associated with it which we sketch below.

 
\begin{center}

\begin{mfpic}[20]{-6}{6}{-5}{5}
\dotted[1pt, 3pt] \polyline{(-5,0), (5,0)}
\dotted[1pt, 3pt] \polyline{(0,-4), (0,4)}
\tlabel[cc](-3,-0.5){$F_{\mbox{\tiny$1$}}$}
\tlabel[cc](3,-0.5){$F_{\mbox{\tiny$2$}}$}
\plotsymbol[3pt]{Asterisk}{(-3,0),(3,0)}
\plotsymbol[3pt]{Cross}{(0,0)}
\tlabel[cc](5.5,0){$V_{\mbox{\tiny$2$}}$}
\tlabel[cc](-5.5,0){$V_{\mbox{\tiny$1$}}$}
\gclear \tlabelrect[cc](0,-0.5){$C$}
\tlabel[cc](2.5,0.5){\tiny Major Axis}
\tlabel[cc](-0.5,2){\tiny \rotatebox{90}{Minor Axis}}
\penwd{1.25pt}
\ellipse{(0,0),5,4}
\point[4pt]{(5,0),(-5,0)}
\end{mfpic}

An ellipse with center $C$; foci $F_{\mbox{\tiny$1$}}$, $F_{\mbox{\tiny$2$}}$; and vertices $V_{\mbox{\tiny$1$}}$, $V_{\mbox{\tiny$2$}}$

\end{center} 

As depicted above,  the \index{ellipse ! center}\index{center ! of an ellipse}\textbf{center} of the ellipse is the midpoint of the line segment connecting the two foci.  The \index{ellipse ! major axis}\index{major axis of an ellipse}\textbf{major axis} of the ellipse is the line segment connecting two opposite ends of the ellipse which also contains the center and foci.  The \index{ellipse ! minor axis}\index{minor axis of an ellipse}\textbf{minor axis} of the ellipse is the line segment connecting two opposite ends of the ellipse which contains the center but is perpendicular to the major axis.  The \index{ellipse ! vertices}\index{vertex ! of an ellipse}\textbf{vertices} of an ellipse are the points of the ellipse which lie on the major axis. 

\smallskip

 Notice that the center is also the midpoint of the major axis, hence it is the midpoint of the vertices. Also note that the major axis is the longer of the two axes through the center, hence the moniker `major.'  Likewise, the minor axis is the shorter of the two, whence the adjective `minor.' 


\smallskip

In order to derive the standard equation of an ellipse, we assume that the ellipse has its center at $(0,0)$, its major axis along the $x$-axis, and has foci $(c,0)$ and $(-c,0)$ and vertices $(-a,0)$ and $(a,0)$.  We will label the $y$-intercepts of the ellipse as $(0,b)$ and $(0,-b)$  (We assume $a$, $b$, and $c$ are all positive numbers.) 

\smallskip

\begin{center}

\begin{mfpic}[20]{-6}{6}{-5}{5}
\axes
\dashed \polyline{(-3,0), (4,2.4)}
\dashed \polyline{(3,0), (4,2.4)}
\tlabel[cc](-3,-0.5){$(-c,0)$}
\tlabel[cc](3,-0.5){$(c,0)$}
\plotsymbol[3pt]{Asterisk}{(-3,0),(3,0)}
\plotsymbol[3pt]{Cross}{(0,0)}
\tlabel[cc](-6,-0.5){$(-a,0)$}
\tlabel[cc](6,-0.5){$(a,0)$}
\tlabel[cc](1,4.5){$(0,b)$}
\tlabel[cc](1, -4.5){$(0,-b)$}
\tlabel[cc](5, 2.4){$(x,y)$}
\tlabel(6,0.25){\scriptsize $x$}
\tlabel(-0.5,5){\scriptsize $y$}
\penwd{1.25pt}
\ellipse{(0,0),5,4}
\point[4pt]{(4,2.4),(-5,0),(5,0),(0,4),(0,-4)}
\end{mfpic}

\end{center}

\smallskip

Note that since $(a,0)$ is on the ellipse, it must satisfy the conditions of Definition \ref{ellipsedefn}.  That is, the distance from $(-c,0)$ to $(a,0)$ plus the distance from  $(c,0)$ to $(a,0)$ must equal the fixed distance $d$.  Since all of these points lie on the $x$-axis, we get

\[ \begin{array}{rclr} \mbox{distance from $(-c,0)$ to $(a,0)$} + \mbox{distance from $(c,0)$ to $(a,0)$} & = & d & \\ (a+c) + (a-c) & = & d & \\ 2a & = & d \\ \end{array}\]

\smallskip

In other words, the fixed distance $d$ mentioned in the definition of the ellipse is none other than the length of the major axis.  We now use that fact $(0,b)$ is on the ellipse, along with the fact that $d=2a$ to get

\[ \begin{array}{rclr} \mbox{distance from  $(-c,0)$ to $(0,b)$} + \mbox{distance from $(c,0)$ to $(0,b)$} & = & 2a & \\ \sqrt{(0-(-c))^2+(b-0)^2} + \sqrt{(0-c)^2+(b-0)^2} & = & 2a & \\ \sqrt{b^2+c^2}+\sqrt{b^2+c^2} & = & 2a  \\ 2 \sqrt{b^2+c^2} & = & 2a \\ \sqrt{b^2+c^2} & = & a \end{array}\]

\smallskip

From this, we get $a^2 = b^2 + c^2$, or $b^2 = a^2 - c^2$, which will prove useful later.  Now consider a point $(x,y)$ on the ellipse.  Applying Definition \ref{ellipsedefn}, we get


\[ \begin{array}{rclr} \mbox{distance from  $(-c,0)$ to $(x,y)$} + \mbox{distance from $(c,0)$ to $(x,y)$} & = & 2a & \\ \sqrt{(x-(-c))^2+(y-0)^2} + \sqrt{(x-c)^2+(y-0)^2} & = & 2a & \\ \sqrt{(x+c)^2+y^2}+\sqrt{(x-c)^2+y^2} & = & 2a  \\ \end{array}\]

\smallskip

In order to make sense of this situation, we need to make good use of Intermediate Algebra.

\[ \begin{array}{rclr}  \sqrt{(x+c)^2+y^2}+\sqrt{(x-c)^2+y^2} & = & 2a  & \\ \sqrt{(x+c)^2+y^2} & = & 2a - \sqrt{(x-c)^2+y^2} & \\ \left(\sqrt{(x+c)^2+y^2}\right)^2 & = & \left(2a - \sqrt{(x-c)^2+y^2}\right)^2 & \\ (x+c)^2+y^2 & = & 4a^2 - 4a\sqrt{(x-c)^2+y^2} + (x-c)^2+y^2 & \\ 4a\sqrt{(x-c)^2+y^2} & = & 4a^2 + (x-c)^2 - (x+c)^2 & \\ 4a\sqrt{(x-c)^2+y^2} & = & 4a^2 - 4cx & \\ a\sqrt{(x-c)^2+y^2} & = & a^2 - cx & \\ \left(a\sqrt{(x-c)^2+y^2}\right)^2 & = & \left(a^2 - cx\right)^2 & \\ a^2\left((x-c)^2+y^2\right) & = & a^4 - 2a^2cx +c^2 x^2 & \\  a^2x^2 - 2a^2cx + a^2c^2+a^2 y^2 & = & a^4 - 2a^2cx +c^2 x^2 & \\  a^2x^2 - c^2 x^2 +a^2 y^2 & = & a^4  - a^2c^2 & \\ \left(a^2 - c^2\right) x^2 +a^2 y^2 & = & a^2 \left(a^2 - c^2\right)  & \\ \end{array}\]

\smallskip

We are nearly finished.  Recall that $b^2 = a^2 - c^2$ so that

\[ \begin{array}{rclr} \left(a^2 - c^2\right)x^2 +a^2 y^2 & = & a^2\left(a^2 - c^2\right) & \\ b^2 x^2 +a^2 y^2 & = & a^2 b^2 & \\  \dfrac{x^2}{a^2} + \dfrac{y^2}{b^2} & = & 1 &  \\ \end{array}\]

\smallskip

This equation is for an ellipse centered at the origin. To get the formula for the ellipse centered at $(h,k)$, we could use the transformations from Section \ref{Transformations} or re-derive the equation using Definition \ref{ellipsedefn} and the distance formula to obtain the formula below.

\smallskip

\colorbox{ResultColor}{\bbm

\begin{eqn}  \label{standardellipse} \index{ellipse ! standard equation} \textbf{The Standard Equation of an Ellipse:}  

For positive unequal numbers $a$ and $b$, the equation of an ellipse with center $(h,k)$ is

\[ \dfrac{(x-h)^2}{a^2} + \dfrac{(y-k)^2}{b^2} = 1 \]

\end{eqn}
  
\ebm}
  
\smallskip

Some remarks about Equation \ref{standardellipse} are in order.  First note that the values $a$ and $b$ determine how far in the $x$ and $y$ directions, respectively, one counts from the center to arrive at points on the ellipse.  

\smallskip

Also  note that if $a > b$, then we have an ellipse whose major axis is horizontal, and hence, the foci lie to the left and right of the center.  In this case, as we've seen in the derivation, the distance from the center to the focus, $c$, can be found by $c = \sqrt{a^2 - b^2}$. 

\smallskip

If $b > a$, the roles of the major and minor axes are reversed, and the foci lie above and below the center. In this case, $c = \sqrt{b^2 - a^2}$.  In either case, it's best to just remember that $c$ is the distance from the center to each focus, and, formulaically,  $c = \sqrt{\mbox{bigger denominator} - \mbox{smaller denominator}}$.  

\smallskip

Finally, it is worth mentioning that if we compare Equation \ref{standardellipse} with the alternate standard equation of the circle, Equation \ref{standardcirclealternate}, the only difference between the forms is that with a circle, the denominators are the same, and with an ellipse, they are different.  

\smallskip

If we take a transformational approach, we can consider both Equations \ref{standardellipse} and \ref{standardcirclealternate} as shifts and stretches of the Unit Circle $x^2 + y^2 = 1$ in Definition \ref{UnitCircle}.  Replacing $x$ with $(x-h)$ and $y$ with $(y-k)$ causes the usual horizontal and vertical shifts.  Replacing $x$ with $\frac{x}{a}$ and $y$ with $\frac{y}{b}$ causes the usual vertical and horizontal stretches. 

\smallskip

 In other words, it is perfectly fine to think of an ellipse as the deformation of a circle in which the circle is stretched farther in one direction than the other.


\smallskip

\begin{ex} \label{ellipsesfirstex} $~$

\begin{enumerate}

\item  Graph each of the following equations below in the $xy$-plane.  Find the center, the lines which contain the major and minor axes, the vertices, the endpoints of the minor axis, and the foci.

\begin{enumerate}

\item  $25(x+1)^2 + 9(y-2)^2 = 225$.  

\item \label{ctsellipseex}  $x^2+4y^2-2x+24y+33 = 0$.

\end{enumerate}

\item  Graph $f(x) = 1 + 2\sqrt{-x^2-4x-3}$

\item  Find the standard form of the equation of an ellipse which satisfies the following characteristics:

\begin{enumerate}

\item  the foci are at $(2,1)$ and $(4,1)$ and vertex $(0,1)$.

\newpage

\item  the ellipse graphed below:

\begin{center}

\begin{mfpic}[20]{-1}{7}{-5}{5}
\axes
\tlabel[cc](7,-0.5){\scriptsize $x$}
\tlabel[cc](0.5,5){\scriptsize $y$}
\tlabel[cc](0.5,0.3){\scriptsize $(0,0)$}
\tlabel[cc](1.5,2.75){\scriptsize $(1,3)$}
\tlabel[cc](5.3,0.3){\scriptsize $(6,0)$}
\xmarks{1,2,3,4,5,6}
\ymarks{-4, -3, -2, -1, 0, 1, 2, 3, 4}
\tlpointsep{4pt}
\scriptsize
\axislabels {x}{   {$1$} 1,  {$2$} 2,  {$3$} 3,  {$4$} 4,  {$5$} 5}
\axislabels {y}{{$-4$} -4,{$-3$} -3,  {$-2$} -2, {$-1$} -1, {$1$} 1, {$2$} 2, {$3$} 3, {$4$} 4}
\normalsize
\penwd{1.25pt}
\ellipse{(3,0),3,4.025}
\point[4pt]{(0,0), (6,0), (1,3)}
\end{mfpic}

\end{center}


\end{enumerate}


\end{enumerate}

\smallskip

{\bf Solution.} 

\begin{enumerate}

\item 

\begin{enumerate}

\item To put   $25(x+1)^2 + 9 (y-2)^2 = 225$ in the form prescribed by Equation \ref{standardellipse}, we rewrite the quantity $(x+1)$ as $(x-(-1))$ and divide through by $225$ to obtain an expression equal to $1$:  

\[  25 (x+1)^2 + 9 (y-2)^2 = 225 \rightarrow \dfrac{(x-(-1))^2}{9}  + \dfrac{(y-2)^2}{25} = 1 \leftrightarrow  \dfrac{(x-(-1))^2}{(3)^2}  + \dfrac{(y-2)^2}{(5)^2} = 1.  \]

We identify $h = -1$ and  $k = 2$, so the center of the ellipse is $(-1,2)$.  We have $a = 3$ so  we move $3$ units left and right from the center to obtain two points on the ellipse:  $(-1-3, 2) = (-4,2)$ and $(-1+3, 2) = (2,2)$.  Likewise, since $b=5$, we move up and down $5$ units from the center to find two more points on the ellipse: $(-1,2+5) = (-1,7)$ and $(-1, 2-5) = (-1,-3)$.

\smallskip

As an aid to sketching, we draw a rectangle matching this description, called a \index{ellipse ! guide rectangle}\index{guide rectangle ! for an ellipse}\textbf{guide rectangle}, and sketch the ellipse inside this rectangle as seen below on the left.

\smallskip

Since we moved farther from the center in the $y$ direction than in the $x$ direction, the major axis will lie along the vertical line $x=-1$, while the minor axis lies along the horizontal line $y = 2$.  The vertices are the points on the ellipse which lie along the major axis so in this case, they are the points $(-1,7)$ and $(-1,-3)$, and the endpoints of the minor axis are $(-4,2)$ and $(2,2)$.  (Notice these points are the four points we used to draw the guide rectangle.)  

\smallskip

To find the foci, we find $c = \sqrt{25-9} = \sqrt{16} = 4$, which means the foci lie $4$ units from the center. Since the major axis is vertical, the foci lie $4$ units above and below the center, at $(-1, 2-4) = (-1,-2)$ and $(-1, 2+4) = (-1,6)$.  Our final graph appears below on the right.

\begin{center}

\begin{tabular}{cc}

\begin{mfpic}[20]{-5}{3}{-4}{8}
\axes
\tlabel(3,-0.5){\scriptsize $x$}
\tlabel(0.5,8){\scriptsize $y$}
\tlabel[cc](-4.75,2){\scriptsize $(-4,2)$}
\tlabel[cc](2.5,2){\scriptsize $(2,2)$}
\tlabel[cc](-1,7.5){\scriptsize $(-1,7)$}
\tlabel[cc](-1,-3.5){\scriptsize $(-1,-3)$}
\tcaption{\scriptsize The graph of $25(x+1)^2 + 9 (y-2)^2 = 225$.}
\xmarks{-4, -3, -2, -1, 0, 1, 2}
\ymarks{-3, -2, -1, 0, 1, 2, 3, 4, 5, 6, 7}

\dotted[1pt, 3pt] \polyline{(-4,-3), (2,-3), (2,7), (-4,7), (-4,-3) }
\plotsymbol[3pt]{Cross}{(-1,2)}

\tlpointsep{4pt}
\scriptsize
\axislabels {x}{{$-3 \hspace{7pt}$} -3, {$-2 \hspace{7pt}$} -2, {$-1 \hspace{7pt}$} -1, {$1$} 1}
\axislabels {y}{  {$-2$} -2, {$-1$} -1, {$1$} 1, {$2$} 2, {$3$} 3, {$4$} 4, {$5$} 5, {$6$} 6}
\normalsize
\penwd{1.25pt}
\ellipse{(-1,2),3,5}
\point[4pt]{(-4,2), (2,2), (-1,7), (-1,-3)}
\end{mfpic}   \hspace{0.75in} & 

\begin{mfpic}[20]{-5}{3}{-4}{8}
\axes
\tlabel(3,-0.5){\scriptsize $x$}
\tlabel(0.5,8){\scriptsize $y$}
\tlabel[cc](-4.75,2){\scriptsize $(-4,2)$}
\tlabel[cc](2.5,2){\scriptsize $(2,2)$}
\tlabel[cc](-1,7.5){\scriptsize $(-1,7)$}
\tlabel[cc](-1,-3.5){\scriptsize $(-1,-3)$}
\tcaption{\scriptsize The graph of $25(x+1)^2 + 9 (y-2)^2 = 225$.}
\xmarks{-4, -3, -2, -1, 0, 1, 2}
\ymarks{-3, -2, -1, 0, 1, 2, 3, 4, 5, 6, 7}
\ellipse{(-1,2),3,5}
\plotsymbol[3pt]{Asterisk}{(-1,-2)}
\plotsymbol[3pt]{Asterisk}{(-1,6)}
\plotsymbol[3pt]{Cross}{(-1,2)}
\point[3pt]{(-1,7)}
\point[3pt]{(-1,-3)}
\dotted[1pt, 3pt] \polyline{(-1,7), (-1,-3)}
\dotted[1pt, 3pt] \polyline{(-4,2), (2,2)}
\tlpointsep{4pt}
\scriptsize
\axislabels {x}{{$-4 \hspace{7pt}$} -4, {$-3 \hspace{7pt}$} -3, {$-2 \hspace{7pt}$} -2,  {$1$} 1,  {$2$} 2}
\axislabels {y}{ {$-2$} -2, {$-1$} -1, {$1$} 1,  {$3$} 3, {$4$} 4, {$5$} 5, {$6$} 6}
\normalsize
\penwd{1.25pt}
\ellipse{(-1,2),3,5}
\point[4pt]{(-4,2), (2,2), (-1,7), (-1,-3)}
\end{mfpic}  \\

\end{tabular}

\end{center}



\item  In the equation $x^2+4y^2-2x+24y+33 = 0$ we have a sum of two squares with unequal coefficients, it's a good bet we have an ellipse on our hands.\footnote{Recall the equation of a parabola has exactly \textit{one} squared variable and the equation of a circle has two squared variables, but  with \textit{identical} coefficients.}  

\smallskip

In order to put this equation into the form stated in Equation \ref{standardellipse}, we need to complete both squares and then divide, if necessary, to get the right-hand side equal to $1$:

\[ \begin{array}{rclr} x^2+4y^2-2x+24y+33 & = & 0 & \\ 

x^2-2x+4y^2+24y & = & -33 & \text{subtract $33$ from both sides} \\   

x^2 - 2x + 4\left(y^2+6y\right)  &  = & - 33 &  \text{factor out leading coefficients} \\   

\left(x^2 - 2x +\underline{1}\right) + 4\left(y^2+6y+\underline{\underline{9}}\right)  &  = & - 33 + \underline{1} + 4(\, \underline{\underline{9}} \, )& \text{complete the squares} \\ 

(x-1)^2 + 4(y+3)^2  &  = & 4 & \text{factor}  \\[5pt]  

\dfrac{(x-1)^2 + 4(y+3)^2}{4}  &  = & \dfrac{4}{4} & \text{divide through by $4$} \\[10pt]

 \dfrac{(x-1)^2}{4} + (y+3)^2 & = & 1 & \\[10pt] 
 
 \dfrac{(x-1)^2}{(2)^2} + \dfrac{(y-(-3))^2}{(1)^2} & = & 1 &  \text{rewrite in the form of Equation \ref{standardellipse}} \\  \end{array} \]

Now that this equation is in the standard form of Equation \ref{standardellipse}, identify $h = 1$ and  $k = -3$ so our ellipse is centered at $(1,-3)$.  With $a=2$,  we move $2$ units left and right from the center to get two points on the ellipse:  $(1-2,-3) = (-1,-3)$ and $(1+2, -3) = (3,-3)$.  Since $b=1$, we move $1$ unit up and down from the center to obtain two additional points on the ellipse:  $(1,-3+1) = (1,-2)$ and $(1,-3-1) = (1,-4)$.  

\smallskip

 Since we moved farther from the center in the $x$ direction than in the $y$ direction, the major axis will lie along the horizontal line $y=-3$ so the minor axis lies along the vertical line $x = 1$. The vertices are the points on the ellipse which lie along the major axis so in this case, they are the points $(-1,-3)$ and $(3,-3)$, and the endpoints of the minor axis are $(1,-2)$ and $(1,-4)$. 
 
 \smallskip
 
 To find the foci, we find $c = \sqrt{4-1} = \sqrt{3}$, which means the foci lie $\sqrt{3}$ units from the center. Since the major axis is horizontal, the foci lie $\sqrt{3}$ units to the left and right of the center, at $(1-\sqrt{3},-3)$ and $(1+\sqrt{3},-3)$.  Plotting all of this information gives the graph below on the left.
 
 \end{enumerate}
 
 \item  At first glance, it doesn't seem as if the function $f(x) = 1 + 2\sqrt{-x^2-4x-3}$ will have \textit{any} graph owing to the presence of the `$-$' signs which decorate \textit{all} of the terms beneath the radical.  However, since $x$ is a \textit{variable}, $-x^2 -4x -3$ is not necessarily negative.\footnote{Indeed, we leave it to the reader to show $-x^2-4x-3 \geq 0$ on $[-3,-1]$.}
 
 
 \smallskip
 
Recall to graph the function $f(x) = 1 + 2\sqrt{-x^2-4x-3}$, we graph the equation $y = 1 + 2\sqrt{-x^2-4x-3}$.  To make sense of this equation in the context of this chapter, we first isolate, then eliminate, the square root in order to obtain a quadratic equation in one or more variables.

\smallskip
 
 From $y = 1 + 2\sqrt{-x^2-4x-3}$, we get $y-1 =  2\sqrt{-x^2-4x-3}$ so that $(y-1)^2 = (2\sqrt{-x^2-4x-3})^2$.  Hence,  $(y-1)^2 = 4(-x^2-4x-3)$ which is equivalent to $4x^2 + 16x + (y-1)^2 = -12$. At this point, we see we have the sum of two squared variables with unequal coefficients present which indicates an ellipse, so we complete the square on $x$ and rewrite the equation so it fits  Equation \ref{standardellipse}:
 
 \[ \begin{array}{rclr} 4x^2 + 16x + (y-1)^2 & = &  -12 & \\ 

4 \left(x^2 + 4x \right)+(y-1)^2  &  = & -12 &  \text{factor out leading coefficient of $x^2$} \\   

4\left(x^2 + 4x +\underline{4}\right) + (y-1)^2  &  = & -12 + 4( \, \underline{4} \, ) & \text{complete the square in $x$} \\ 

4(x+2)^2 + (y-1)^2  &  = & 4 & \text{factor}  \\[5pt]  

\dfrac{4(x+2)^2 + (y-1)^2}{4}  &  = & \dfrac{4}{4} & \text{divide through by $4$} \\[10pt]

(x+2)^2 + \dfrac{(y-1)^2}{4} & = & 1 & \\[10pt] 
 
 \dfrac{(x-(-2))^2}{(1)^2} + \dfrac{(y-1)^2}{(2)^2} & = & 1 &  \text{rewrite in the form of Equation \ref{standardellipse}} \\  \end{array} \]
 
 We identify $h=-2$ and $k=1$ so the ellipse is centered at $(-2,1)$.  With $a=1$, we move $1$ unit to the left and to the right an obtain the points $(-2-1,1) = (-3,1)$ and $(-2+1, 1) = (-1,1)$.  With $b=2$, we move $2$ units up and down from the center to obtain two more points on the graph of the ellipse: $(-2, 1+2) = (-2,3)$ and $(-2, 1-2) = (-2,-1)$. 
 
 \smallskip
 
However, the graph of $f(x) = 1 + 2\sqrt{-x^2-4x-3}$  cannot be the \textit{entire} ellipse, else it would violate the vertical line test.  This means the graph of $f$ must be a \textit{portion} of the ellipse.  Since, by definition $\sqrt{-x^2-4x-3} \geq 0$, we know $f(x) =  1 + 2\sqrt{-x^2-4x-3} \geq 1$.  Hence, the graph of  $f$ must be the \textit{upper} half of the ellipse, as denoted below on the right.

\begin{center}

\begin{tabular}{cc}

\begin{mfpic}[25]{-2}{5}{-5}{1}
\axes
\tlabel(5,-0.25){\scriptsize $x$}
\tlabel(0.25, 1){\scriptsize $y$}
\xmarks{-1,0,1,2,3,4}
\ymarks{-4,-3,-2,-1,0}
\tlabel[cc](4,-3){\scriptsize $(3,-3)$}
\tlabel[cc](-2,-3){\scriptsize $(-1,-3)$}
\tlabel[cc](1,-1.5){\scriptsize $(1,-2)$}
\tlabel[cc](1,-4.5){\scriptsize $(1,-4)$}
\tcaption{\scriptsize The graph of $x^2+4y^2-2x+24y+33 = 0$.}
\dotted[1pt, 3pt] \polyline{(-1,-3),(3,-3)}
\dotted[1pt, 3pt] \polyline{(1,-2),(1,-4)}
\plotsymbol[3pt]{Asterisk}{(2.732050808,-3)}
\plotsymbol[3pt]{Asterisk}{(-0.732050808,-3)}
\plotsymbol[3pt]{Cross}{(1,-3)}
\tlpointsep{4pt}
\scriptsize
\axislabels {x}{{$-1 \hspace{6pt}$} -1,{$1$} 1, {$2$} 2, {$3$} 3, {$4$} 4}
\axislabels {y}{{$-4$} -4,  {$-2$} -2, {$-1$} -1,}
\normalsize
\penwd{1.25pt}
\ellipse{(1,-3),2,1}
\point[4pt]{(3,-3), (-1,-3), (1,-2), (1,-4)}
\end{mfpic}

&

\begin{mfpic}[25]{-4}{2}{-2}{4}
\axes
\tlabel[cc](2,-0.5){\scriptsize $x$}
\tlabel[cc](0.5, 4){\scriptsize $y$}
\xmarks{-3,-2,-1,1}
\ymarks{-1,2,3}
\tlabel[cc](-3.75,1){\scriptsize $(-3,1)$}
\gclear \tlabelrect(-0.25,1){\scriptsize $(-1,1)$ \vphantom{$\left( \dfrac{1}{2} \right)$}}
\tlabel[cc](-2,3.25){\scriptsize $(-2,3)$}
\tcaption{\scriptsize The graph of $f(x) = 1 + 2\sqrt{-x^2-4x-3}$.}
\dotted \ellipse{(-2,1),1,2}
\tlpointsep{4pt}
\scriptsize
\axislabels {x}{ {$-3 \hspace{7pt}$} -3,{$-2 \hspace{7pt}$} -2,{$-1 \hspace{7pt}$} -1,{$1$} 1}
\axislabels {y}{ {$-1$} -1,{$2$} 2,{$3$} 3 }
\normalsize
\penwd{1.25pt}
\parafcn{0, 3.14 , 0.1}{(cos(t)-2, (2*sin(t))+1)}
\point[4pt]{(-3,1), (-1,1), (-2,3)}
\end{mfpic} \\

\end{tabular}

\end{center}


\item  \begin{enumerate}

\item  We plot the information given to us below and notice immediately that the major axis is horizontal, which means $a > b$.   Since the center is the midpoint of the foci, we know the center of the ellipse is  $(3, 1)$ which means  $h = 3$ and $k = 1$. Since one vertex is at $(0,1)$, which is $3$ units from the center,  we have that $a = 3$, so $a^2 =  9$.   At this point, all that remains is to find $b^2$. 


\begin{center}

\begin{mfpic}[25]{-1}{6}{0}{2}
\axes
\xmarks{ 0,1,2,3,4,5}
\ymarks{0,1}
\point[3pt]{(0,1)}
\plotsymbol[3pt]{Asterisk}{(2,1)}
\plotsymbol[3pt]{Asterisk}{(4,1)}
\tlabel(6, -0.25){\scriptsize $x$}
\tlabel(0.25,2){\scriptsize $y$}
\tlpointsep{4pt}
\scriptsize
\axislabels {x}{{$1$} 1, {$2$} 2, {$3$} 3, {$4$} 4, {$5$} 5}
\axislabels {y}{{$1$} 1}
\normalsize
\end{mfpic}

\end{center}

Since the foci, $(2,1)$ and $(4,1)$,  are $1$ unit away from the center, we have $c=1$.  Putting this together with the fact that  $a > b$, we get $c = \sqrt{a^2-b^2}$, or $1 = \sqrt{9-b^2}$.  Squaring both sides gives $1 = 9-b^2$ or   $b^2 = 8$.  Feeding all of this data into Equation \ref{standardellipse}, we get our final answer:  \[\dfrac{(x-3)^2}{9}+\dfrac{(y-1)^2}{8}=1.\]

\item From the diagram, we infer the ellipse is taller than it is wide.  More specifically, the labeled points $(0,0)$ and $(6,0)$ are the endpoints of the minor axis.  This gives the center is $(3,0)$, so $h = 3$ and $k = 0$.  Moreover, we have $a = 3$.  


\smallskip


While it certainly \textit{appears} that the vertices are $(3, \pm 4)$, in which case we'd have $b = 4$, these points aren't labeled.  Instead, we use the labeled point $(1,3)$ to calculate $b^2$.  At this stage, we know the equation of the ellipse is \[ \dfrac{(x-3)^2}{9} + \dfrac{y^2}{b^2} = 1,\] so upon substituting $x=1$ and $y=3$, we obtain $\frac{4}{9} + \frac{9}{b^2} = 1$.  Solving this equation, we get $b^2 = \frac{81}{5}$.  Hence, our final answer is:

 \[\dfrac{(x-3)^2}{9}+\dfrac{5(y-1)^2}{81}=1.\]

Note the vertices of the ellipse are $\left(3, \pm \sqrt{\frac{81}{5}} \right) \approx (3, \pm 4)$ but they are not \textit{exactly} $(3, \pm 4)$.  \qed

\end{enumerate}

\end{enumerate}
 
\end{ex}

As seen in Example \ref{ellipsesfirstex} above, it is often necessary to algebraically manipulate a given equation into the standard form of Equation \ref{standardellipse} in order to graph.  We summarize one approach below.

\smallskip

\colorbox{ResultColor}{\bbm

\centerline{\textbf{To Write the Equation of an Ellipse in Standard Form}}

\begin{enumerate}

\item  Group common variables together on one side of the equation and put the constant on the other.

\item  Complete the square on both variables as needed.

\item  Divide both sides, if needed,  to obtain $1$ on one side of the equation.

\end{enumerate}

\ebm}

\smallskip


If we think of a circle as being `perfectly round,' then ellipses, being deformed circles, have varying degrees of `roundness.'   We quantify this idea with the notion of \textit{eccentricity} defined formally below.

\smallskip

\colorbox{ResultColor}{\bbm

\begin{defn}

The \label{ellipseeccentricity} \index{ellipse ! eccentricity} \index{eccentricity} \textbf{eccentricity} of an ellipse, denoted $e$, is the following ratio:

\[  e = \dfrac{\mbox{distance from the center to a focus}}{\mbox{distance from the center to a vertex}} \]

\end{defn}

\ebm}

\smallskip

In an ellipse, the foci are closer to the center than the vertices, so $0 < e < 1$. The ellipse below on left has eccentricity  $e \approx 0.66$;  for the ellipse below on the right, $e \approx 0.98$.  In general, the closer the eccentricity is to $0$, the less `eccentric' or more  `circular' the ellipse appears.   On the other hand, the closer the eccentricity is to $1$, the more `eccentric' the ellipse is and it appears less `circular.'

\begin{center}

\begin{tabular}{cc}

\begin{mfpic}[10]{-5}{5}{-3}{3}
\penwd{1.25pt}
\ellipse{(0,0),4,3}
\end{mfpic} 
  &

\hspace{1in} \begin{mfpic}[10]{-5}{5}{-3}{3}
\penwd{1.25pt}
\ellipse{(0,0),5,1}
\end{mfpic} \\

 $e \approx 0.66$

&

\hspace{1in} $e \approx 0.98$ \\

\end{tabular}

\end{center}

According to \href{https://en.wikipedia.org/wiki/Kepler's_laws_of_planetary_motion}{\underline{Kepler's Laws of Planetary Motion}}, each planet orbits the Sun in an elliptical path  with the Sun at one focus.  The eccentricity is therefore an important orbital parameter.  We investigate the orbit of Mercury in the following example.

\smallskip

\begin{ex}  \label{MecuryOribitEx} According to \href{http://nssdc.gsfc.nasa.gov/planetary/factsheet/planet_table_british.html}{\underline{NASA}}, Mercury orbits the Sun in an elliptical orbit.  If at perihelion,\footnote{the closest Mercury is to the Sun} Mercury is  28.6 megamiles from the Sun and at aphelion,\footnote{the farthest Mercury is from the Sun} Mercury is 43.4 megamiles from the Sun, find the eccentricity of Mercury's orbit, rounded to three decimal places.

\smallskip

{\bf Solution.}  Per Kepler's Laws, the orbit of Mercury is an ellipse with the Sun at one focus.  Since we are told to assume the Sun is positioned at $(0,0)$, we are free to choose if the major axis of the ellipse lies on the $x$- or $y$-axis.  We choose the former.

\smallskip

With the information given, we know one vertex is 28.6 units away from the focus and the other is 43.4 units.  Again, we are free to chose which direction is which, so we decide to put one vertex at $(-28.6, 0)$ and the other is $(43.4, 0)$.  Schematically, we have:

\begin{center}

\begin{mfpic}[15]{-6}{6}{-2}{2}
\axes
\tlabel(6,-0.5){\scriptsize $x$}
\tlabel(0.5,1){\scriptsize $y$}
\xmarks{-5,-4,-3,-2,-1,0,1,2,3,4,5}
\ymarks{-1,1}
\tlabel[cc](-2.86, 0.5){\scriptsize $(-28.6, 0)$}
\tlabel[cc](4.34, 0.5){\scriptsize $(43.4, 0)$}
\plotsymbol[4pt]{Asterisk}{(0,0)}
\point[4pt]{(-2.86,0),(4.34,0)}
\tlpointsep{4pt}
\scriptsize
\axislabels {x}{ {$-50 \hspace{7pt}$} -5,{$-30 \hspace{7pt}$} -3,{$-10 \hspace{7pt}$} -1,{$10$} 1,{$30$} 3,{$50$} 5}
\end{mfpic}

\end{center}

Since the center is the midpoint of the vertices, we find the center to be $(7.4, 0)$.  This means the focus $(0,0)$ is $7.4$ units from the center and each vertex is $36$ units from the center. This is precisely what we need to determine the eccentricity of Mercury's orbit:

\[   e = \dfrac{\mbox{distance from the center to a focus}}{\mbox{distance from the center to a vertex}}  = \dfrac{7.4}{36} \approx 0.205. \]

To our delight, we find answer agrees with  \href{http://nssdc.gsfc.nasa.gov/planetary/factsheet/planet_table_british.html}{\underline{NASA}}. With an eccentricity of $0.205$, we expect Mercury's orbit to be fairly `round.'   In Exercise \ref{MercuryOrbitGraph}, we invite the reader to find the equation of Mercury's orbit to confirm this conclusion graphically.  \qed

\end{ex}

As with parabolas, ellipses have a  \index{ellipse ! reflective property} reflective property. If we imagine the dashed lines below representing sound waves, then it can be shown that the waves emanating from one focus reflect off the top of the ellipse and head towards the other focus.  

\smallskip

\begin{center}

\begin{mfpic}[20]{-6}{6}{-5}{5}
\dashed \polyline{(-3,0), (4,2.4)}
\dashed \polyline{(3,0), (4,2.4)}
\tlabel[cc](-3,-0.5){$F_{\mbox{\tiny$1$}}$}
\tlabel[cc](3,-0.5){$F_{\mbox{\tiny$2$}}$}
\plotsymbol[3pt]{Asterisk}{(-3,0),(3,0)}
\penwd{1.25pt}
\parafcn{0,3.14,.1}{(5*cos(t),4*sin(t))}
\point[4pt]{(4,2.4)}
\end{mfpic}

\end{center}

Such geometry is exploited in the construction of so-called `Whispering Galleries'. If a person whispers at one focus, a person standing at the other focus will hear the first person as if they were standing right next to them.  We explore the Whispering Galleries in our last example. 

\begin{ex} \label{whisgalleryex} Jamie and Jason want to exchange secrets (terrible secrets) from across a crowded whispering gallery.  Recall that a whispering gallery is a room which, in cross section, is half of an ellipse.  If the room is 40 feet high at the center and 100 feet wide at the floor, how far from the outer wall should each of them stand so that they will be positioned at the foci of the ellipse? 

\smallskip

{\bf Solution.} Ultimately, we are looking for information about the relative position of foci and the vertices, so we assume the whispering gallery can be represented by the upper half of an ellipse centered at the origin with its major axis along the $x$-axis. Graphing the data yields the schematic below.

\begin{center}

\begin{mfpic}[25]{-6}{6}{-1}{5}
\axes
\tlabel(6,-0.5){\scriptsize $x$}
\tlabel(0.5,5){\scriptsize $y$}

\arrow \reverse \arrow \polyline{(-5,-0.5),(5,-0.5)}
\gclear \tlabelrect[cc](0,-1.5){$100$ units wide}
\arrow \reverse \arrow \polyline{(0,0.25),(0,3.75)}
\gclear \tlabelrect[cc](0, 2){$40$ units tall}
\penwd{1.25pt}
\parafcn{0,pi,0.1}{(5*cos t, 4*sin t)}
\end{mfpic}
\end{center}

Since the ellipse is $100$ units wide and $40$ units tall, we get $a=50$ and $b=40$, respectively.  From this, we get $c = \sqrt{50^2-40^2} = \sqrt{900} = 30$, which means the foci are $30$ units from the center.  Hence, the foci are $50-30=20$ units from the vertices so Jason and Jamie should stand $20$ feet from opposite ends of the gallery to exchange their secrets in what amounts to very public privacy. \qed

\end{ex}

\newpage

\subsection{Exercises}

\label{ExercisesforEllipses}

In Exercises \ref{graphellipseexfirst} - \ref{graphellipseexlast},  graph the ellipse in the $xy$-plane.  Find the center, the lines which contain the major and minor axes, the vertices, the endpoints of the minor axis, the foci and the eccentricity.

\begin{multicols}{2}
\begin{enumerate}

\item $\dfrac{x^{2}}{169} + \dfrac{y^{2}}{25} = 1$  \label{graphellipseexfirst} \label{oddellipseone}
\item $\dfrac{x^2}{9} + \dfrac{y^2}{25} = 1$



\setcounter{HW}{\value{enumi}}
\end{enumerate}
\end{multicols}

\begin{multicols}{2}
\begin{enumerate}
\setcounter{enumi}{\value{HW}}

\item $\dfrac{(x - 2)^{2}}{4} + \dfrac{(y + 3)^{2}}{9} = 1$   \label{oddellipsethree}
\item $\dfrac{(x + 5)^{2}}{16} + \dfrac{(y - 4)^{2}}{1} = 1$


\setcounter{HW}{\value{enumi}}
\end{enumerate}
\end{multicols}

\begin{multicols}{2}
\begin{enumerate}
\setcounter{enumi}{\value{HW}}

\item $\dfrac{(x - 1)^{2}}{10} + \dfrac{(y - 3)^{2}}{11} = 1$   \label{oddellipsefive}
\item $\dfrac{(x-1)^2}{9}+\dfrac{(y+3)^2}{4} = 1$


\setcounter{HW}{\value{enumi}}
\end{enumerate}
\end{multicols}

\begin{multicols}{2}
\begin{enumerate}
\setcounter{enumi}{\value{HW}}

\item $\dfrac{(x+2)^2}{16}+\dfrac{(y-5)^2}{20} = 1$   \label{oddellipseseven}
\item $\dfrac{(x-4)^2}{8} + \dfrac{(y-2)^2}{18} = 1$ \label{graphellipseexlast}

\setcounter{HW}{\value{enumi}}
\end{enumerate}
\end{multicols}

In Exercises \ref{stdformellipseexfirst} - \ref{stdformellipseexlast}, put the equation in standard form.   Find the center, the lines which contain the major and minor axes, the vertices, the endpoints of the minor axis, the foci and the eccentricity.\footnote{\ldots assuming the equation were graphed in the $xy$-plane.}

\begin{multicols}{2}
\begin{enumerate}
\setcounter{enumi}{\value{HW}}

\item $9x^2+25y^2-54x-50y-119=0$  \label{stdformellipseexfirst}   \label{oddellipsenine}
\item $12x^{2} + 3y^{2} - 30y + 39 = 0$

\setcounter{HW}{\value{enumi}}
\end{enumerate}
\end{multicols}

\begin{multicols}{2}
\begin{enumerate}
\setcounter{enumi}{\value{HW}}

\item $5x^{2} + 18y^{2} - 30x + 72y + 27 = 0$   \label{oddellipseeleven}
\item $x^2 - 2x + 2y^2 - 12y + 3 = 0$


\setcounter{HW}{\value{enumi}}
\end{enumerate}
\end{multicols}

\begin{multicols}{2}
\begin{enumerate}
\setcounter{enumi}{\value{HW}}

\item $9x^2 + 4y^2 - 4y - 8 = 0$   \label{oddellipsethirteen}
\item $6x^2+5y^2-24x+20y+14=0$  \label{stdformellipseexlast}

\setcounter{HW}{\value{enumi}}
\end{enumerate}
\end{multicols}

\begin{enumerate}
\setcounter{enumi}{\value{HW}}

\item For each of the odd numbered equations given in Exercises \ref{oddellipseone} - \ref{oddellipsethirteen}, find two or more explicit functions of $x$ represented by each of the equations.  (See Example \ref{horizontalparabolaex} in Section \ref{Parabolas}.)

\setcounter{HW}{\value{enumi}}
\end{enumerate}

In Exercises \ref{semiellipsefunctionfirst} - \ref{semiellipsefunctionlast}, graph each function by recognizing it as a semi ellipse.

\begin{multicols}{2}
\begin{enumerate}
\setcounter{enumi}{\value{HW}}

\item   $f(x) = \sqrt{16-4x^2}$ \label{semiellipsefunctionfirst}
\item   $g(x) = -\frac{1}{2} \sqrt{6x-x^2}$

\setcounter{HW}{\value{enumi}}
\end{enumerate}
\end{multicols}

\begin{multicols}{2}
\begin{enumerate}
\setcounter{enumi}{\value{HW}}

\item  $f(x) = -2\sqrt{3-2x-x^2}$
\item  $g(x) = -2 + 2\sqrt{9-x^2}$ \label{semiellipsefunctionlast}

\setcounter{HW}{\value{enumi}}
\end{enumerate}
\end{multicols}

\pagebreak

In Exercises \ref{buildellipsefromgraphfirst} - \ref{buildellipsefromgraphlast}, find an equation for the ellipse or semi ellipse whose graph is given.

\begin{multicols}{2}
\begin{enumerate}
\setcounter{enumi}{\value{HW}}

\item $~$ \label{buildellipsefromgraphfirst}  % $\dfrac{(x-1)^2}{9} + \dfrac{y^2}{16} = 1$

\begin{mfpic}[13]{-4}{6}{-5}{5}
\axes
\tlabel[cc](6,-0.5){\scriptsize $x$}
\tlabel[cc](0.5,5){\scriptsize $y$}
\tlabel[cc](1, 4.5){\scriptsize $(1,4)$}
\tlabel[cc](1.25, -4.75){\scriptsize $(1,-4)$}
\tlabel[cc](-3, 0.75){\scriptsize $(-2,0)$}
\tlabel[cc](5, 0.75){\scriptsize $(4,0)$}
\xmarks{-3 step 1 until 5}
\ymarks{-4 step 1 until 4}
\tlpointsep{4pt}
\scriptsize
\axislabels {x}{ {$-3 \hspace{7pt}$} -3,  {$-1 \hspace{7pt}$} -1, {$1$} 1, {$2$} 2, {$3$} 3, {$5$} 5}
\axislabels {y}{ {$-4$} -4, {$-2$} -2, {$-1$} -1, {$1$} 1, {$2$} 2,  {$4$} 4  }
\penwd{1.25pt}
\ellipse{(1,0), 3,4}
\point[4pt]{(-2,0), (4,0), (1,4), (1,-4)}
\normalsize
\end{mfpic} 

\vfill

\columnbreak

\item $~$  % $\dfrac{(x-4)^2}{16} + \dfrac{(y-4)^2}{9} = 1$

\begin{mfpic}[13]{-1}{9}{-1}{9}
\axes
\tlabel[cc](9,-0.5){\scriptsize $x$}
\tlabel[cc](0.5,9){\scriptsize $y$}
\tlabel[cc](1, 4){\scriptsize $(0,4)$}
\tlabel[cc](7, 4){\scriptsize $(8,4)$}
\tlabel[cc](4, 7.75){\scriptsize $(4,7)$}
\tlabel[cc](4, 0.25){\scriptsize $(4,1)$}
\xmarks{1 step 1 until 8}
\ymarks{1 step 1 until 8}
\tlpointsep{4pt}
\scriptsize
\axislabels {x}{{$1$} 1, {$2$} 2, {$3$} 3, {$4$} 4, {$5$} 5, {$6$} 6, {$7$} 7, {$8$} 8}
\axislabels {y}{{$1$} 1, {$2$} 2, {$3$} 3, {$4$} 4, {$5$} 5, {$6$} 6, {$7$} 7, {$8$} 8}
\penwd{1.25pt}
\ellipse{(4,4), 4, 3}
\point[4pt]{(4,1), (0,4), (4,7), (8,4)}
\normalsize
\end{mfpic} 

\setcounter{HW}{\value{enumi}}
\end{enumerate}
\end{multicols}


\begin{multicols}{2}
\begin{enumerate}
\setcounter{enumi}{\value{HW}}


\item $~$   % $y = 3 - \frac{3}{4} \sqrt{16-x^2}$

\begin{mfpic}[13]{-5}{5}{-1}{6}
\axes
\tlabel[cc](5,-0.5){\scriptsize $x$}
\tlabel[cc](0.5,6){\scriptsize $y$}
\tlabel[cc](-4, 3.5){\scriptsize $(-4,3)$}
\tlabel[cc](4, 3.5){\scriptsize $(4,3)$}
\tlabel[cc](0.75, -0.75){\scriptsize $(0, 0)$}
%\tlabel[cc](-0.5,-1){\scriptsize $\left(0, \frac{1}{2} \right)$}
\xmarks{-4,-3,-2,-1,1,2,3,4}
\ymarks{1 step 1 until 5}
\tlpointsep{4pt}
\scriptsize
\axislabels {x}{ {$-4 \hspace{7pt}$} -4, {$-3 \hspace{7pt}$} -3, {$-2 \hspace{7pt}$} -2, {$-1 \hspace{7pt}$} -1,  {$4$} 4,  {$3$} 3,  {$2$} 2}
\axislabels {y}{{$1$} 1, {$2$} 2, {$3$} 3,  {$4$} 4,  {$5$} 5}
\penwd{1.25pt}
\function{-4,4,0.1}{3-0.75*sqrt(16-(x**2))}
\point[4pt]{(-4,3), (0,0), (4,3)}
%\tcaption{ \scriptsize $x$,$y$-intercept $(0,0)$}
\normalsize
\end{mfpic} 

\vfill

\columnbreak

\item $~$ \label{buildellipsefromgraphlast}   % $y = 2 \sqrt{8x-x^2-12}$

\begin{mfpic}[13]{-1}{9}{-1}{6}
\axes
\tlabel[cc](9,-0.5){\scriptsize $x$}
\tlabel[cc](0.5,6){\scriptsize $y$}
\tlabel[cc](6, -0.75){\scriptsize $(6,0)$}
\tlabel[cc](4, 4.5){\scriptsize $(4,4)$}
\tlabel[cc](2, -0.75){\scriptsize $(2, 0)$}
%\tlabel[cc](-0.5,-1){\scriptsize $\left(0, \frac{1}{2} \right)$}
\xmarks{1 step 1 until 8}
\ymarks{1 step 1 until 5}
\tlpointsep{4pt}
\scriptsize
\axislabels {x}{{$1$} 1,   {$3$} 3,  {$4$} 4,  {$5$} 5,  {$8$} 8, {$7$} 7 }
\axislabels {y}{{$1$} 1, {$2$} 2, {$3$} 3,  {$4$} 4,  {$5$} 5}
\penwd{1.25pt}
\function{2,6,0.1}{2*sqrt(8*x-12-(x**2))}
\point[4pt]{(6,0), (2,0), (4,4)}
%\tcaption{ \scriptsize $x$,$y$-intercept $(0,0)$}
\normalsize
\end{mfpic} 


\setcounter{HW}{\value{enumi}}
\end{enumerate}
\end{multicols}


In Exercises \ref{buildellipsefirst} - \ref{buildellipselast},  find the standard form of the equation of the ellipse which has the given properties.

\begin{enumerate}
\setcounter{enumi}{\value{HW}}

\item Center $(3, 7)$, Vertex $(3, 2)$, Focus $(3, 3)$  \label{buildellipsefirst}
\item Foci $(0, \pm 5)$, Vertices $(0, \pm 8)$.
\item Foci $(\pm 3, 0)$, length of the Minor Axis $10$
\item Vertices $(3,2)$, $(13,2)$; Endpoints of the Minor Axis $(8,4)$, $(8,0)$
\item Center $(5,2)$, Vertex $(0,2)$, eccentricity $\frac{1}{2}$
\item All points on the ellipse are in Quadrant IV except $(0, -9)$ and $(8, 0)$.  (One might also say that the ellipse is ``tangent to the axes'' at those two points.)  \label{buildellipselast}

\setcounter{HW}{\value{enumi}}
\end{enumerate}

\begin{enumerate}
\setcounter{enumi}{\value{HW}}

\item  Repeat Example \ref{whisgalleryex} for a whispering gallery 200 feet wide and 75 feet tall.

\item \label{ellipsearchex} An elliptical arch is constructed which is 6 feet wide at the base and 9 feet tall in the middle. Find the height of the arch exactly 1 foot in from the base of the arch. Compare your result with your answer to Exercise \ref{parabolaarch} in Section \ref{Parabolas}.

\item The Earth's orbit around the sun is an ellipse with the sun at one focus and eccentricity $e \approx 0.0167$.  The length of the semimajor axis (that is, half of the major axis) is defined to be $1$ astronomical unit (AU).  The vertices of the elliptical orbit are given special names: `aphelion' is the vertex farthest from the sun, and  `perihelion' is the vertex closest to the sun.  Find the distance in AU between the sun and aphelion and the distance in AU between the sun and perihelion.

\item  \label{MercuryOrbitGraph}  This exercise is a follow-up to Example \ref{MecuryOribitEx}.  Find the equation of the ellipse which models the orbit of Mercury.  Graph the ellipse using a graphing utility, and comment on the `roundness' of the orbit.


\item Some famous examples of whispering galleries include \href{http://www.stpauls.co.uk/}{\underline{St. Paul's Cathedral}} in London, England, \href{http://www.aoc.gov/cc/capitol/nat_stat_hall.cfm}{\underline{National Statuary Hall}} in Washington, D.C.,  and \href{http://www.cincymuseum.org/}{\underline{The Cincinnati Museum Center}}. With the help of your classmates, research these whispering galleries.  How does the whispering effect compare and contrast with the scenario in Example  \ref{whisgalleryex}?

\item With the help of your classmates, research ``extracorporeal shock-wave lithotripsy''.  It uses the reflective property of the ellipsoid to dissolve kidney stones.

\end{enumerate}

\newpage

\subsection{Answers}

\begin{enumerate}

\item \begin{multicols}{2} \raggedcolumns
$\dfrac{x^{2}}{169} + \dfrac{y^{2}}{25} = 1$

Center $(0, 0)$\\
Major axis along $y = 0$\\
Minor axis along $x = 0$\\
Vertices $(13, 0), \, (-13, 0)$\\
Endpoints of Minor Axis $(0,-5)$, $(0,5)$ \\
Foci $(12, 0), \, (-12, 0)$\\
$e = \frac{12}{13}$\\

\begin{mfpic}[7][10]{-14}{14}{-6}{6}
\axes
\tlabel(14,-0.5){\scriptsize $x$}
\tlabel(0.5,6){\scriptsize $y$}
\xmarks{-13 step 1 until 13}
\ymarks{-5 step 1 until 5}
\plotsymbol[4pt]{Asterisk}{(-12,0), (12,0)}
\plotsymbol[4pt]{Cross}{(0,0)}
\point[4pt]{(-13,0), (13,0), (0,5), (0,-5)}
\tlpointsep{4pt}
\tiny
\axislabels {x}{{$-13 \hspace{6pt}$} -13, {$-1 \hspace{6pt}$} -1, {$1$} 1, {$13$} 13}
\axislabels {y}{{$-5$} -5, {$-4$} -4, {$-3$} -3, {$-2$} -2, {$-1$} -1, {$1$} 1, {$2$} 2, {$3$} 3, {$4$} 4, {$5$} 5}
\normalsize
\penwd{1.25pt}
\ellipse{(0,0),13,5}
\end{mfpic} 

\end{multicols}

\item \begin{multicols}{2} \raggedcolumns
$\dfrac{x^{2}}{9} + \dfrac{y^{2}}{25} = 1$

Center $(0, 0)$\\
Major axis along $x = 0$\\
Minor axis along $y = 0$\\
Vertices $(0,5), \, (0,-5)$\\
Endpoints of Minor Axis $(-3,0)$, $(3,0)$ \\
Foci $(0,-4), \, (0,4)$\\
$e = \frac{4}{5}$\\

\begin{mfpic}[10]{-4}{4}{-6}{6}
\axes
\tlabel(4,-0.5){\scriptsize $x$}
\tlabel(0.5,6){\scriptsize $y$}
\xmarks{-3 step 1 until 3}
\ymarks{-5 step 1 until 5}
\plotsymbol[4pt]{Asterisk}{(0,-4), (0,4)}
\plotsymbol[4pt]{Cross}{(0,0)}
\point[4pt]{(-3,0), (3,0), (0,5), (0,-5)}
\tlpointsep{4pt}
\tiny
\axislabels {x}{{$-3 \hspace{6pt}$} -3, {$-2 \hspace{6pt}$} -2, {$-1 \hspace{6pt}$} -1,{$1$} 1, {$2$} 2, {$3$} 3}
\axislabels {y}{{$-5$} -5, {$-4$} -4, {$-3$} -3, {$-2$} -2, {$-1$} -1, {$1$} 1, {$2$} 2, {$3$} 3, {$4$} 4, {$5$} 5}
\normalsize
\penwd{1.25pt}
\ellipse{(0,0),3,5}
\end{mfpic} 

\end{multicols}

\item \begin{multicols}{2} \raggedcolumns
$\dfrac{(x - 2)^{2}}{4} + \dfrac{(y + 3)^{2}}{9} = 1$

Center $(2, -3)$\\
Major axis along $x = 2$\\
Minor axis along $y = -3$\\
Vertices $(2, 0), \, (2, -6)$\\
Endpoints of Minor Axis $(0,-3)$, $(4,-3)$\\
Foci $(2, -3 + \sqrt{5}), \, (2, -3 - \sqrt{5})$\\
$e = \frac{\sqrt{5}}{3}$\\

\begin{mfpic}[15]{-1}{5}{-7}{1}
\axes
\tlabel(5,-0.5){\scriptsize $x$}
\tlabel(0.5,1){\scriptsize $y$}
\xmarks{1 step 1 until 4}
\ymarks{-6 step 1 until 0}
\plotsymbol[4pt]{Asterisk}{(2, -0.7639), (2,-5.23606)}
\plotsymbol[4pt]{Cross}{(2,-3)}
\point[4pt]{(2,0), (2,-6), (0,-3), (4,-3)}
\tlpointsep{4pt}
\scriptsize
\axislabels {x}{{$1$} 1, {$2$} 2, {$3$} 3, {$4$} 4}
\axislabels {y}{{$-6$} -6, {$-5$} -5, {$-4$} -4, {$-3$} -3, {$-2$} -2, {$-1$} -1}
\normalsize
\penwd{1.25pt}
\ellipse{(2,-3),2,3}
\end{mfpic} 

\end{multicols}

\item \begin{multicols}{2} \raggedcolumns
$\dfrac{(x + 5)^{2}}{16} + \dfrac{(y - 4)^{2}}{1} = 1$

Center $(-5, 4)$\\
Major axis along $y = 4$\\
Minor axis along $x = -5$\\
Vertices $(-9, 4), \, (-1, 4)$\\
Endpoints of Minor Axis $(-5,3)$, $(-5,5)$\\
Foci $(-5 + \sqrt{15}, 4), \, (-5 - \sqrt{15}, 4)$\\
$e = \frac{\sqrt{15}}{4}$\\

\begin{mfpic}[16]{-10}{1}{-.5}{6}
\axes
\tlabel(1,-0.5){\scriptsize $x$}
\tlabel(0.5,6){\scriptsize $y$}
\xmarks{-9 step 1 until -1}
\ymarks{1 step 1 until 5}
\plotsymbol[4pt]{Asterisk}{(-8.873,4), (-1.127,4)}
\plotsymbol[4pt]{Cross}{(-5,4)}
\point[4pt]{(-9,4), (-1,4), (-5,5), (-5,3)}
\tlpointsep{4pt}
\scriptsize
\axislabels {x}{{$-9 \hspace{7pt}$} -9, {$-8 \hspace{7pt}$} -8, {$-7 \hspace{7pt}$} -7, {$-6 \hspace{7pt}$} -6, {$-5 \hspace{7pt}$} -5, {$-4 \hspace{7pt}$} -4, {$-3 \hspace{7pt}$} -3, {$-2 \hspace{7pt}$} -2, {$-1 \hspace{7pt}$} -1}
\axislabels {y}{{$1$} 1, {$2$} 2, {$3$} 3, {$4$} 4, {$5$} 5}
\normalsize
\penwd{1pt}
\ellipse{(-5,4),4,1}
\end{mfpic} 

\end{multicols}

\item \begin{multicols}{2} \raggedcolumns
$\dfrac{(x - 1)^{2}}{10} + \dfrac{(y - 3)^{2}}{11} = 1$

Center $(1, 3)$\\
Major axis along $x = 1$\\
Minor axis along $y = 3$\\
Vertices $(1, 3 + \sqrt{11}), \, (1, 3 - \sqrt{11})$\\
Endpoints of the Minor Axis \\ $(1-\sqrt{10}, 3), \, (1+\sqrt{10}, 3)$\\
Foci $(1, 2), \, (1, 4)$\\
$e = \frac{\sqrt{11}}{11}$\\

\begin{mfpic}[18]{-3}{5}{-1}{7}
\axes
\tlabel(5,-0.25){\scriptsize $x$}
\tlabel(0.25,7){\scriptsize $y$}
\xmarks{-2 step 1 until 4}
\ymarks{1 step 1 until 6}
\plotsymbol[4pt]{Asterisk}{(1,2), (1,4)}
\plotsymbol[4pt]{Cross}{(1,3)}
\point[4pt]{(1,6.3166), (1,-0.3166), (-2.1623, 3), (4.1623,3)}
\tlpointsep{4pt}
\scriptsize
\axislabels {x}{{$-2 \hspace{7pt}$} -2, {$-1 \hspace{7pt}$} -1, {$1$} 1, {$2$} 2, {$3$} 3, {$4$} 4}
\axislabels {y}{{$1$} 1, {$2$} 2, {$3$} 3, {$4$} 4, {$5$} 5, {$6$} 6}
\normalsize
\penwd{1.25pt}
\ellipse{(1,3),3.1623,3.3166}
\end{mfpic} 

\end{multicols}

\item \begin{multicols}{2} \raggedcolumns
$\dfrac{(x-1)^2}{9}+\dfrac{(y+3)^2}{4} = 1$

Center $(1, -3)$\\
Major axis along $y = -3$\\
Minor axis along $x = 1$\\
Vertices $(4, -3), \, (-2, -3)$\\
Endpoints of the Minor Axis $(1,-1)$, $(1,-5)$\\
Foci $(1+\sqrt{5}, -3), \, (1-\sqrt{5}, -3)$\\
$e = \frac{\sqrt{5}}{3}$\\

\begin{mfpic}[18]{-3}{5}{-6}{1}
\axes
\tlabel(5,-0.25){\scriptsize $x$}
\tlabel(0.25,1){\scriptsize $y$}
\xmarks{-2 step 1 until 4}
\ymarks{-5 step 1 until -1}
\plotsymbol[4pt]{Asterisk}{(3.2361,-3), (-1.2361,-3)}
\plotsymbol[4pt]{Cross}{(1,-3)}
\point[4pt]{(4,-3), (-2,-3), (1,-1), (1,-5)}
\tlpointsep{4pt}
\scriptsize
\axislabels {x}{{$-2 \hspace{7pt}$} -2, {$-1 \hspace{7pt}$} -1, {$1$} 1, {$2$} 2, {$3$} 3, {$4$} 4}
\axislabels {y}{{$-5$} -5, {$-4$} -4, {$-3$} -3, {$-2$} -2, {$-1$} -1}
\normalsize
\penwd{1.25pt}
\ellipse{(1,-3),3,2}
\end{mfpic} 

\end{multicols}



\item \begin{multicols}{2} \raggedcolumns
$\dfrac{(x+2)^2}{16}+\dfrac{(y-5)^2}{20} = 1$

Center $(-2, 5)$\\
Major axis along $x = -2$\\
Minor axis along $y = 5$\\
Vertices $(-2, 5 + 2\sqrt{5}), \, (-2, 5 - 2\sqrt{5})$\\
Endpoints of the Minor Axis $(-6,5)$, $(2,5)$
Foci $(-2, 7), \, (-2, 3)$\\
$e = \frac{\sqrt{5}}{5}$\\

\begin{mfpic}[18]{-7}{3}{-1}{11}
\axes
\tlabel(3,-0.25){\scriptsize $x$}
\tlabel(0.25,11){\scriptsize $y$}
\xmarks{-6 step 1 until 2}
\ymarks{1 step 1 until 11}
\plotsymbol[4pt]{Asterisk}{(-2,7), (-2,3)}
\plotsymbol[4pt]{Cross}{(-2,5)}
\point[4pt]{(-2,9.4721), (-2,0.5279), (-6,5), (2,5)}
\tlpointsep{4pt}
\scriptsize
\axislabels {x}{{$-6 \hspace{7pt}$} -6,{$-5 \hspace{7pt}$} -5,{$-4 \hspace{7pt}$} -4,{$-3 \hspace{7pt}$} -3,{$-2 \hspace{7pt}$} -2, {$-1 \hspace{7pt}$} -1, {$1$} 1, {$2$} 2}
\axislabels {y}{{$1$} 1, {$2$} 2, {$3$} 3, {$4$} 4, {$5$} 5, {$6$} 6, {$7$} 7, {$8$} 8, {$9$} 9, {$10$} 10 }
\normalsize
\penwd{1.25pt}
\ellipse{(-2,5),4,4.4721}
\end{mfpic} 

\end{multicols}


\pagebreak


\item \begin{multicols}{2} \raggedcolumns
$\dfrac{(x-4)^2}{8}+\dfrac{(y-2)^2}{18} = 1$

Center $(4, 2)$\\
Major axis along $x = 4$\\
Minor axis along $y = 2$\\
Vertices $(4, 2 + 3\sqrt{2}), \, (4, 2 - 3\sqrt{2})$\\
Endpoints of the Minor Axis \\
$(4-2\sqrt{2},2)$, $(4+2\sqrt{2},2)$\\
Foci $(4, 2+\sqrt{10}), \, (4, 2-\sqrt{10})$\\
$e = \frac{\sqrt{5}}{3}$\\

\begin{mfpic}[18]{-1}{8}{-4}{8}
\axes
\tlabel(8,-0.25){\scriptsize $x$}
\tlabel(0.25,8){\scriptsize $y$}
\xmarks{1 step 1 until 7}
\ymarks{-3 step 1 until 7}
\plotsymbol[4pt]{Asterisk}{(4,5.1623), (4,-1.1623)}
\plotsymbol[4pt]{Cross}{(4,2)}
\point[4pt]{(4,6.2426), (4,-2.2426), (1.1716, 2), (6.828,2)}
\tlpointsep{4pt}
\scriptsize
\axislabels {x}{{$1$} 1, {$2$} 2, {$3$} 3, {$4$} 4, {$5$} 5, {$6$} 6, {$7$} 7}
\axislabels {y}{{$-3$} -3,{$-2$} -2,{$-1$} -1,{$1$} 1, {$2$} 2, {$3$} 3, {$4$} 4, {$5$} 5, {$6$} 6, {$7$} 7}
\normalsize
\penwd{1.25pt}
\ellipse{(4,2),2.8284,4.2426}
\end{mfpic} 

\end{multicols}

\setcounter{HW}{\value{enumi}}
\end{enumerate}

\begin{multicols}{2}
\begin{enumerate}
\setcounter{enumi}{\value{HW}}

\item  $\dfrac{(x-3)^2}{25} + \dfrac{\left(y-1\right)^2}{9} = 1$\\
Center  $\left(3, 1 \right)$\\
Major Axis along $y=1$\\
Minor Axis along $x=3$\\
Vertices  $\left( 8, 1   \right)$, $(-2, 1)$\\
Endpoints of Minor Axis $\left(3,4\right)$, $\left(3,-2\right)$\\
Foci $\left(7,1 \right)$, $\left(-1, 1\right)$\\
$e = \frac{4}{5}$


\vfill

\columnbreak


\item $\dfrac{x^{2}}{3} + \dfrac{(y - 5)^{2}}{12} = 1$\\
Center $(0, 5)$\\
Major axis along $x = 0$\\
Minor axis along $y = 5$\\
Vertices $(0, 5 - 2\sqrt{3}), (0, 5 + 2\sqrt{3})$\\
Endpoints of Minor Axis $(-\sqrt{3},5)$, $(\sqrt{3},5)$\\
Foci $(0, 2), (0, 8)$\\
$e = \frac{\sqrt{3}}{2}$


\setcounter{HW}{\value{enumi}}
\end{enumerate}
\end{multicols}

\begin{multicols}{2}
\begin{enumerate}
\setcounter{enumi}{\value{HW}}

\item $\dfrac{(x - 3)^{2}}{18} + \dfrac{(y + 2)^{2}}{5} = 1$\\
Center $(3, -2)$\\
Major axis along $y = -2$\\
Minor axis along $x = 3$\\
Vertices $(3 - 3\sqrt{2}, -2), (3 + 3\sqrt{2}, -2)$\\
Endpoints of Minor Axis $(3,-2+\sqrt{5})$, $(3,-2-\sqrt{5})$\\
Foci $(3 - \sqrt{13}, -2), (3 + \sqrt{13}, -2)$\\
$e = \frac{\sqrt{26}}{6}$


\vfill

\columnbreak

\item $\dfrac{(x - 1)^{2}}{16} + \dfrac{(y - 3)^{2}}{8} = 1$\\
Center  $(1,3)$ \\
Major Axis along $y=3$\\
Minor Axis along $x=1$\\
Vertices $(5, 3)$, $(-3,3)$\\
Endpoints of Minor Axis $(1,3+2\sqrt{2})$, $(1,3-2\sqrt{2})$\\
Foci $(1 + 2 \sqrt{2}, 3)$, $(1-2 \sqrt{2},3)$\\
$e = \frac{\sqrt{2}}{2}$


\setcounter{HW}{\value{enumi}}
\end{enumerate}
\end{multicols}

\pagebreak

\begin{multicols}{2}
\begin{enumerate}
\setcounter{enumi}{\value{HW}}

\item  $\dfrac{x^2}{1} + \dfrac{4\left(y-\frac{1}{2}\right)^2}{9} = 1$\\
Center  $\left(0, \frac{1}{2} \right)$\\
Major Axis along $x=0$ (the $y$-axis)\\
Minor Axis along $y=\frac{1}{2}$\\
Vertices  $\left( 0, 2   \right)$, $(0, -1)$\\
Endpoints of Minor Axis $\left(-1, \frac{1}{2} \right)$, $\left(1, \frac{1}{2} \right)$\\
Foci $\left(0, \frac{1+\sqrt{5}}{2}\right)$, $\left(0, \frac{1-\sqrt{5}}{2}\right)$\\
$e = \frac{\sqrt{5}}{3}$

\vfill

\columnbreak


\item  $\dfrac{(x-2)^2}{5} + \dfrac{\left(y+2\right)^2}{6} = 1$\\
Center  $\left(2, -2 \right)$\\
Major Axis along $x=2$\\
Minor Axis along $y=-2$\\
Vertices  $\left( 2, -2+\sqrt{6}   \right)$, $(2, -2-\sqrt{6})$\\
Endpoints of Minor Axis $\left(2-\sqrt{5},-2 \right)$, $\left(2+\sqrt{5},-2\right)$\\
Foci $\left(2,-1 \right)$, $\left(2, -3\right)$\\
$e = \frac{\sqrt{6}}{6}$



\setcounter{HW}{\value{enumi}}
\end{enumerate}
\end{multicols}



\begin{enumerate}
\setcounter{enumi}{\value{HW}}

\item $~$
For number \ref{oddellipseone}:

\begin{itemize}

\item  $f(x) = \frac{5}{13} \sqrt{169-x^2}$ represents the upper half of the ellipse.

\item  $g(x) =  -\frac{5}{13} \sqrt{169-x^2}$   represents the lower half of the ellipse.

\end{itemize}


For number \ref{oddellipsethree}:

\begin{itemize}

\item  $f(x) = -3 + \frac{3}{2} \sqrt{4x-x^2}$ represents the upper half of the ellipse.

\item  $g(x) =   -3 - \frac{3}{2} \sqrt{4x-x^2}$   represents the lower half of the ellipse.

\end{itemize}


For number \ref{oddellipsefive}:

\begin{itemize}

\item  $f(x) = 3 + \frac{1}{10} \sqrt{990+220x-110x^2}$ represents the upper half of the ellipse.

\item  $g(x) =   3 - \frac{1}{10} \sqrt{990+220x-110x^2}$   represents the lower half of the ellipse.

\end{itemize}


For number \ref{oddellipseseven}:

\begin{itemize}

\item  $f(x) = 5 + \frac{1}{2} \sqrt{60-20x-5x^2}$ represents the upper half of the ellipse.

\item  $g(x) =  5 - \frac{1}{2} \sqrt{60-20x-5x^2}$  represents the lower half of the ellipse.

\end{itemize}


For number \ref{oddellipsenine}:

\begin{itemize}

\item  $f(x) = 1 + \frac{3}{5} \sqrt{16+6x-x^2}$ represents the upper half of the ellipse.

\item  $g(x) =  1 - \frac{3}{5} \sqrt{16+6x-x^2} $   represents the lower half of the ellipse.

\end{itemize}


For number \ref{oddellipseeleven}:

\begin{itemize}

\item  $f(x) = -2 + \frac{1}{6} \sqrt{90+60x-10x^2}$ represents the upper half of the ellipse.

\item  $g(x) =  -2 -  \frac{1}{6} \sqrt{90+60x-10x^2}  $   represents the lower half of the ellipse.

\end{itemize}

\newpage


For number \ref{oddellipsethirteen}:

\begin{itemize}

\item  $f(x) = \frac{1}{2} + \frac{3}{2} \sqrt{1-x^2} $ represents the upper half of the ellipse.

\item  $g(x) =  \frac{1}{2} - \frac{3}{2} \sqrt{1-x^2} $   represents the lower half of the ellipse.

\end{itemize}


\setcounter{HW}{\value{enumi}}
\end{enumerate}

\begin{multicols}{2}
\begin{enumerate}
\setcounter{enumi}{\value{HW}}

\item $f(x) = \sqrt{16-4x^2}$

\begin{mfpic}[15]{-4}{4}{-1}{5}
\axes
\tlabel[cc](4,-0.5){\scriptsize $x$}
\tlabel[cc](0.5,5){\scriptsize $y$}
\tlabel[cc](-2, -0.5){\scriptsize $(-2,0)$}
\tlabel[cc](2, -0.5){\scriptsize $(2, 0)$}
\tlabel[cc](-1, 4.5){\scriptsize $(0,4)$}
\xmarks{-3 step 1 until 3}
\ymarks{0 step 1 until 4}
\tlpointsep{4pt}
\scriptsize
%\axislabels {x}{ {$-3 \hspace{7pt}$} -3,  {$1$} 1,  {$3$} 3}
\axislabels {y}{ {$1$} 1, {$2$} 2, {$3$} 3}
\penwd{1.25pt}
\function{-2,2,0.1}{sqrt(16-4*(x**2))}
\point[4pt]{(-2,0), (0,4), (2,0)}
\normalsize
\end{mfpic} 

\vfill

\columnbreak

\item $g(x) = -\frac{1}{2} \sqrt{6x-x^2}$

\begin{mfpic}[15]{-1}{7}{-4.5}{1.5}
\axes
\tlabel[cc](7,-0.5){\scriptsize $x$}
\tlabel[cc](0.5,1.5){\scriptsize $y$}
\tlabel[cc](3, -2.25){\scriptsize $\left(3,-\frac{3}{2} \right)$}
\tlabel[cc](-0.5, 0.75){\scriptsize $(0,0)$}
\tlabel[cc](6, 0.5){\scriptsize $(6,0)$}
\xmarks{1 step 1 until 6}
\ymarks{-4 step 1 until -1}
\tlpointsep{4pt}
\scriptsize
\axislabels {x}{{$1$} 1, {$2$} 2, {$3$} 3, {$4$} 4, {$5$} 5}
\axislabels {y}{{$-1$} -1, {$-2$} -2, {$-3$} -3, {$-4$} -4}
\penwd{1.25pt}
\function{0, 6, 0.1}{-0.5*sqrt(6*x-(x**2))}
\point[4pt]{(0,0), (3,-1.5), (6,0)}
\normalsize
\end{mfpic} 

\setcounter{HW}{\value{enumi}}
\end{enumerate}
\end{multicols}



\begin{multicols}{2}
\begin{enumerate}
\setcounter{enumi}{\value{HW}}

\item  $f(x) = -2\sqrt{3-2x-x^2}$

\begin{mfpic}[20]{-4}{4}{-6}{2}
\axes
\tlabel[cc](4,-0.5){\scriptsize $x$}
\tlabel[cc](0.5,2){\scriptsize $y$}
\tlabel[cc](-3, 0.5){\scriptsize $(-3,0)$}
\tlabel[cc](-1.25, -4.5){\scriptsize $(-1,-4)$}
\tlabel[cc](1, 0.5){\scriptsize $(1,0)$}
\xmarks{-3 step 1 until 3}
\ymarks{-5 step 1 until 1}
\tlpointsep{4pt}
\scriptsize
\axislabels {x}{ {$-1 \hspace{7pt}$} -1, {$-2 \hspace{7pt}$} -2, {$2$} 2,{$3$} 3}
\axislabels {y}{ {$1$} 1, {$-2$} -2,{$-3$} -3,  {$-1$} -1,  {$-5$} -5}
\penwd{1.25pt}
\function{-3,1,0.1}{-2*sqrt(3-2*x-(x**2))}
\point[4pt]{(-3,0), (-1,-4), (1,0)}
\normalsize
\end{mfpic} 



\item  $g(x) = -2 + 2\sqrt{9-x^2}$

\begin{mfpic}[20]{-4}{4}{-3}{5}
\axes
\tlabel[cc](4,-0.5){\scriptsize $x$}
\tlabel[cc](0.5,5){\scriptsize $y$}
\tlabel[cc](1,4.25){\scriptsize $(0,4)$}
\tlabel[cc](-3, -2.5){\scriptsize $(-3,-2)$}
\tlabel[cc](3, -2.5){\scriptsize $(3,-2)$}
\xmarks{-3 step 1 until 3}
\ymarks{-2 step 1 until 4}
\tlpointsep{4pt}
\scriptsize
\axislabels {x}{ {$-1 \hspace{7pt}$} -1, {$-2 \hspace{7pt}$} -2, {$-3 \hspace{7pt}$} -3, {$2$} 2, {$3$} 3, {$1$} 1}
\axislabels {y}{ {$2$} 2,{$3$} 3,{$1$} 1, {$-1$} -1, {$-2$} -2}
\penwd{1.25pt}
\function{-3,3,0.1}{-2+2*sqrt(9-(x**2))}
\point[4pt]{(0,4), (-3,-2), (3,-2)}
\normalsize
\end{mfpic} 

\setcounter{HW}{\value{enumi}}
\end{enumerate}
\end{multicols}


\begin{multicols}{2}
\begin{enumerate}
\setcounter{enumi}{\value{HW}}

\item  $\dfrac{(x-1)^2}{9} + \dfrac{y^2}{16} = 1$

\item $\dfrac{(x-4)^2}{16} + \dfrac{(y-4)^2}{9} = 1$

\setcounter{HW}{\value{enumi}}
\end{enumerate}
\end{multicols}

\begin{multicols}{2}
\begin{enumerate}
\setcounter{enumi}{\value{HW}}

\item $y = 3 - \frac{3}{4} \sqrt{16-x^2}$

\item  $y = 2 \sqrt{8x-x^2-12}$ \vphantom{ $y = 3 - \frac{3}{4} \sqrt{16-x^2}$}

\setcounter{HW}{\value{enumi}}
\end{enumerate}
\end{multicols}

\begin{multicols}{2}
\begin{enumerate}
\setcounter{enumi}{\value{HW}}

\item $\dfrac{(x - 3)^{2}}{9} + \dfrac{(y - 7)^{2}}{25} = 1$
\item $\dfrac{x^{2}}{39} + \dfrac{y^{2}}{64} = 1$


\setcounter{HW}{\value{enumi}}
\end{enumerate}
\end{multicols}



\begin{multicols}{2}
\begin{enumerate}
\setcounter{enumi}{\value{HW}}

\item $\dfrac{x^{2}}{34} + \dfrac{y^{2}}{25} = 1$
\item $\dfrac{(x - 8)^{2}}{25} + \dfrac{(y - 2)^{2}}{4} = 1$

\setcounter{HW}{\value{enumi}}
\end{enumerate}
\end{multicols}

\begin{multicols}{2}
\begin{enumerate}
\setcounter{enumi}{\value{HW}}

\item $\dfrac{(x-5)^{2}}{25} + \dfrac{4(y-2)^{2}}{75} = 1$
\item $\dfrac{(x - 8)^{2}}{64} + \dfrac{(y + 9)^{2}}{81} = 1$

\setcounter{HW}{\value{enumi}}
\end{enumerate}
\end{multicols}

\begin{enumerate}
\setcounter{enumi}{\value{HW}}

\item  Jamie and Jason should stand $100-25\sqrt{7} \approx 33.86$ feet from opposite ends of the gallery.

\item  The arch can be modeled by the upper half of $\frac{x^2}{9} + \frac{y^2}{81} = 1$.  One foot in from the base of the arch corresponds to either $x = \pm 2$.  Plugging in $x = \pm 2$ gives $y = \pm 3\sqrt{5}$ and since $y$ represents a height, we choose $y=3\sqrt{5} \approx 6.71$ feet. 

\item Distance from the sun to aphelion $\approx 1.0167$ AU.\\
Distance from the sun to perihelion $\approx 0.9833$ AU.

\item  $\dfrac{(x-7.4)^2}{1296} + \dfrac{y^2}{1241.24} = 1$.  Graphing this equation\footnote{\ldots using the `Zoom Square' setting \ldots} reveals a very `round' orbit.

\end{enumerate}


\closegraphsfile