\documentclass[11pt]{book}
%\oddsidemargin 0in
%\evensidemargin 0in
%\marginparwidth 0in
%\textheight 8in
%\textwidth 6in
%\topmargin 0in
\headheight 14pt
\usepackage{amssymb,amsmath,amsthm,fancyhdr,supertabular,longtable,hhline,mathtools}
\usepackage{colortbl}
\usepackage{color}
\usepackage{tikz}
%\usepackage{pict2e}
\usepackage{import, multicol,boxedminipage}
\usepackage{chapterfolder}
\usepackage[metapost,truebbox]{mfpic}
%\usepackage[pdflatex]{graphicx}
\usepackage{makeidx}
%\usepackage[colorlinks, hyperindex, plainpages=false, linkcolor=blue, urlcolor=blue, pdfpagelabels]{hyperref}
%\usepackage[all]{hypcap}
%\usepackage{bm}

%\newtheoremstyle{definition}  % this prevents the text in definitions, theorems, and corollaries from being italicized
%\newtheorem{defn}{\bf Definition}[chapter]
%\newtheorem{thm}{\bf Theorem}[chapter]
%\newtheorem{cor}[thm]{\bf Corollary}
%\newtheorem{eqn}{\bf Equation}[chapter]
%\newtheorem{ex}{\bf Example}[section]
\newtheorem{fig}{\bf Figure}[chapter]


\usepackage{mathrsfs,amsfonts, amssymb, amsthm, amsmath, mdframed, paralist}


%\renewcommand{\qedsymbol}{{\scshape QED}}
%\renewenvironment{proof}{{\noindent{\scshape Proof}}}{\mbox{} \hfill \qedsymbol}

%\newtheorem{remark}{Remark}[section]

%\mdfdefinestyle{examplestyle}{%
%     linecolor=black,linewidth=0.5pt,%  
%skipabove={1.2\baselineskip},
 % innertopmargin={1.2\baselineskip},
  %ntheorem=true,
  %hidealllines=true,
  %backgroundcolor=gray!50,
  %splittopskip=2\baselineskip,
     %frametitlerule=true,%
     %frametitlebackgroundcolor=gray!20,
%     innertopmargin=0pt,
 %  }


\usepackage[many]{tcolorbox}

\definecolor{myblue}{RGB}{0,163,243}
\definecolor{myex}{HTML}{ 5c6bc0}
\definecolor{mythm}{HTML}{c0ca33}
\definecolor{mydefn}{HTML}{00acc1}
\definecolor{myeqn}{HTML}{827717}
\definecolor{myinfo}{HTML}{fdd835}


\tcbset{mystyle/.style={
  breakable,
  enhanced,
  outer arc=0pt,
  arc=0pt,
  colframe=myex,
  colback=myex!20,
  attach boxed title to top left,
  leftrule=2pt,bottomrule=2pt,
  boxed title style={
    colback=myex,
    outer arc=0pt,
    arc=0pt,
    %leftrule=2pt,rightrule=2pt,toprule=2pt,bottomrule=2pt,
    },
  title=Example~\thetcbcounter,
  }
}


\newtcolorbox[auto counter, number within=section]{ex}[1][]{
  mystyle,
  colback=white,
  rightrule=0pt,
  toprule=0pt,
  coltitle=white,
  title=\textbf{Example \thetcbcounter},
  }


%\usepackage[breakable,skins]{tcolorbox}

%\newtcolorbox{ex}[1][]{%
 %   breakable,
 %   enhanced,
 %   colback=white,
 %   colframe=blue,
%#1}%

\newtcolorbox{boxinfo}[1][]{%
    %breakable,
    enhanced,  
    outer arc=0pt,
    arc=0pt,
    colback=myinfo!10,
    colframe=myinfo,
#1}%

\newtcolorbox[auto counter,number within=section]{thm}[1][]{
%before=\bigskip\centering,
%after=\bigskip,
%breakable,
enhanced,
outer arc=0pt,
arc=0pt,
colframe=mythm,
colback=mythm!10,
title=\textbf{Theorem ~\thetcbcounter},
sharp corners,
%rounded corners=northeast,
attach boxed title to top left,
boxed title style={
    enhanced,
    colframe=mythm,
    colback=mythm,
    arc=0pt,
    bottomrule=0pt,
    rightrule=0pt,
    sharp corners,
    rounded corners=northeast,
},
interior style={},
frame style={},
overlay unbroken and first={
    \node[anchor=west,color=black] 
    at (title.east) {{\textbf{#1}}};}}

\newtcolorbox[auto counter,number within=section]{defn}[1][]{
%before=\bigskip\centering,
%after=\bigskip,
%breakable,
enhanced,
outer arc=0pt,
arc=0pt,
colframe=mydefn,
colback=mydefn!10,
title=\textbf{Definition ~\thetcbcounter},
sharp corners,
%rounded corners=northeast,
attach boxed title to top left,
boxed title style={
    enhanced,
    colframe=mydefn,
    colback=mydefn,
    arc=0pt,
    bottomrule=0pt,
    rightrule=0pt,
    sharp corners,
    rounded corners=northeast,
},
interior style={},
frame style={},
overlay unbroken and first={
    \node[anchor=west,color=black] 
    at (title.east) {{\textbf{#1}}};}}

\newtcolorbox[auto counter,number within=section]{eqn}[1][]{
%before=\bigskip\centering,
%after=\bigskip,
%breakable,
enhanced,
outer arc=0pt,
arc=0pt,
colframe=myeqn,
colback=myeqn!10,
title=\textbf{Equation ~\thetcbcounter},
sharp corners,
%rounded corners=northeast,
attach boxed title to top left,
boxed title style={
    enhanced,
    colframe=myeqn,
    colback=myeqn,
    arc=0pt,
    bottomrule=0pt,
    rightrule=0pt,
    sharp corners,
    rounded corners=northeast,
},
interior style={},
frame style={},
overlay unbroken and first={
    \node[anchor=west,color=black] 
    at (title.east) {{\textbf{#1}}};}}

%\theoremstyle{definition}
%\newmdtheoremenv[style=examplestyle]{ex}{Example}[section]
%\renewcommand{\theex}{\arabic{chapter}.\arabic{section}.\arabic{example}}


\definecolor{ResultColor}{RGB}{255,255,255}
\setlength{\fboxrule}{0pt}
\setlength{\parindent}{0in}
%\newcommand{\bbm}{\begin{minipage}{6.41in}}
\newcommand{\bbm}{\begin{minipage}{6.91in}}
\newcommand{\ebm}{\end{minipage}}

\makeatletter
\newenvironment{identity}
    {\endgroup\ignorespaces}
    {\begingroup\def\@currenvir{identity}\ignorespacesafterend}
\makeatother

%\newcommand{\bbm}{\begin{empty}}
%\newcommand{\ebm}{\end{empty}}

\usepackage{array}
\setlength{\extrarowheight}{2pt}
\allowdisplaybreaks[2]
\usepackage{cancel}
\usepackage{sectsty}
%\usepackage{appendix}
\usepackage{textcomp}
\usepackage{multirow}
\usepackage[nottoc]{tocbibind}
\usepackage{verbatim}
%\usepackage[margin=0.75in]{geometry}
\usepackage[left=0.75in, right=0.75in, top=0.65in, bottom=0.3in]{geometry}

\DeclareSymbolFont{AMSb}{U}{msb}{m}{n}
\DeclareMathSymbol{\C}{\mathbin}{AMSb}{"43}
\DeclareMathSymbol{\N}{\mathbin}{AMSb}{"4E}
\DeclareMathSymbol{\I}{\mathbin}{AMSb}{"5A}
\DeclareMathSymbol{\Q}{\mathbin}{AMSb}{"51}
\DeclareMathSymbol{\R}{\mathbin}{AMSb}{"52}
\DeclareMathSymbol{\W}{\mathbin}{AMSb}{"57}

\allsectionsfont{\mdseries \scshape}
\makeatletter
\renewcommand\l@section{\@dottedtocline{1}{1.5em}{3em}}
\renewcommand\l@subsection{\@dottedtocline{2}{4.5em}{3.5em}}
\makeatother
\pagestyle{fancy}
\newcounter{HW}
\newcounter{HWindent}

\renewcommand{\textinterrobang}{$! \! \! ?$}

%comment out
\newcommand{\co}[1]{}

%Below is for Iowna Font
%\renewcommand*\sfdefault{iwona}
%\usepackage[math]{iwona}

%Below is for Helvetica (scaled): 
%\usepackage[scaled=.92]{helvet}   
%\renewcommand{\familydefault}{\sfdefault}  %makes the text of the book sans serif
%\usepackage[helvet]{sfmath}  %makes the math in the book sans serif
%\allsectionsfont{\sffamily}  %makes the chapter and section titles sans serif

\usepackage{hyperref}
\hypersetup{
    colorlinks=true,
    linkcolor=blue,
    filecolor=blue,
    urlcolor=blue,
    citecolor=blue
}
\urlstyle{same}

\makeatletter
\newcases{mycases}{\quad}{%
  \hfil$\m@th\displaystyle{##}$}{$\m@th\displaystyle{##}$\hfil}{\lbrace}{.}
\makeatother

\makeindex

\begin{document}

%removed \sc command from lines below.
\renewcommand{\chaptermark}[1]%
                  {\markboth{#1}{}}
\renewcommand{\sectionmark}[1]%
               {\markright{\thesection\ #1}}
\renewcommand{\headrulewidth}{0pt}
\lhead[\fancyplain{}{\thepage}]%
      {\fancyplain{}{\nouppercase{\rightmark}}}
\rhead[\fancyplain{}{\nouppercase{\leftmark}}]%
      {\fancyplain{}{\thepage}}
\cfoot{}




\frontmatter

%removed \sc command from lines below.

\title{\Huge {\sc Functions}}

\author{\small mainly based on Precalculus, Version $4 - \epsilon$, by Carl Stitz, Ph.D. \&  Jeff Zeager, Ph.D.\\
\small some parts based on Beginning and Intermediate Algebra, by Tyler Wallace}


\maketitle

%\include{acknowledgements}

\thispagestyle{empty}

\renewcommand{\contentsname}{Table of Contents}

\addtocontents{toc}{\protect\thispagestyle{empty}}

\clearpage

\pdfbookmark[1]{\contentsname}{toc}

\tableofcontents

%\chapter{The Old Preface}
%\label{OldPreface}
%\thispagestyle{empty}
%\input{OldPreface}

%\chapter{The New Preface}
%\label{NewPreface}
%\thispagestyle{empty}
%\input{NewPreface}

\mainmatter
 
\renewcommand{\chaptername}{Chapter}

\chapter{Basic Concepts and Review}
\thispagestyle{empty}
\section{Basic Set Theory and Interval Notation}
\import{./AlgebraAppendix/}{AppSetTheory}
\newpage
\section{Real Number Arithmetic and Simplifying Exponential Expressions}
\import{./AlgebraAppendix/}{AppRealNumberArithmetic}

\chapter{Introduction to Functions}
\label{IntroductiontoFunctions}
\thispagestyle{empty}
\import{./IntroductiontoFunctions/}{IntroductiontoFunctions}

%\chapter{Polynomial Functions}
%\label{PolynomialFunctions}
%\thispagestyle{empty}
%\import{./PolynomialFunctions/}{PolynomialFunctions}

%\chapter{Rational Functions}
%\label{RationalFunctions}
%\thispagestyle{empty}
%\import{./RationalFunctions/}{RationalFunctions}

\chapter{Root and Radical Functions}
\label{RootRadicalPowerFunctions}
\thispagestyle{empty}
\import{./RootRadicalPowerFunctions/}{RootRadicalPowerFunctions}

\chapter{Further Topics on Functions}
\label{FurtherTopicsonFunctions}
\thispagestyle{empty}
\import{./FurtherTopicsonFunctions/}{FurtherTopicsonFunctions}

\chapter{Exponential and Logarithmic Functions}
\label{ExponentialandLogarithmicFunctions}
\thispagestyle{empty}
\import{./ExponentialandLogarithmicFunctions/}{ExponentialandLogarithmicFunctions}

%\chapter{The Conic Sections}
%\label{TheConicSections}
%\thispagestyle{empty}
%\import{./TheConicSections/}{TheConicSections}

%\chapter{Systems of Equations and Matrices}
%\label{SystemsofEquationsandMatrices}
%\thispagestyle{empty}
%\import{./SystemsofEquationsandMatrices/}{SystemsofEquationsandMatrices}

%\chapter{Sequences and the Binomial Theorem}
%\label{SequencesandtheBinomialTheorem}
%\thispagestyle{empty}
%\import{./SequencesandtheBinomialTheorem/}{SequencesandtheBinomialTheorem}

%\chapter{Foundations of Trigonometry}
%\label{FoundationsofTrigonometry}
%\thispagestyle{empty}
%\import{./FoundationsofTrigonometry/}{FoundationsofTrigonometry}

%\chapter{Analytical Trigonometry}
%\label{AnalyticalTrigonometry}
%\thispagestyle{empty}
%\import{./AnalyticalTrigonometry/}{AnalyticalTrigonometry}

%\chapter{Geometric Applications of Trigonometry}
%\label{GeometricApplicationsofTrigonometry}
%\thispagestyle{empty}
%\import{./GeometricApplicationsofTrigonometry/}{GeometricApplicationsofTrigonometry}

%\chapter{Polar Coordinates and Parametric Equations}
%\label{PolarCoordinatesandParametricEquations}
%\thispagestyle{empty}
%\import{./PolarCoordinatesandParametricEquations/}{PolarCoordinatesandParametricEquations}

%\appendix

%\chapter{Algebra Review}
%\label{AlgebraAppendix}
%\thispagestyle{empty}
%\import{./AlgebraAppendix/}{AlgebraAppendix}

%\chapter{Geometry Review}
%\label{GeometryAppendix}
%\thispagestyle{empty}
%\import{./GeometryAppendix/}{GeometryAppendix}

%\chapter{Selected Proofs}
%\label{ProofsAppendix}
%\thispagestyle{empty}
%\import{./ProofsAppendix/}{ProofsAppendix}

\backmatter

\let\originalstyle=\thispagestyle            % Store the command for later reuse.
%\def\thispagestyle#1{\fancyfoot[C]{}}       % This clears footer in the center if fancyhdr is in use.
\def\thispagestyle#1{\originalstyle{empty}} % Use this to get blank header+footer, TeXnically it is only \thispagestyle{empty}.
%\def\thispagestyle#1{}                       % This line completely ignores the content of the \thispagestyle command.
\printindex                                  % Typeset the actual Index.
\let\thispagestyle=\originalstyle            % Let's get back to the original version of the \thispagestyle, if needed later in the document.



\end{document}
