\noindent Let me start the Preface to \underline{Precalculus}: Episode IV - A New Approach, by explaining why there were no additional prefaces between this one and our first one written on January 25, 2010.  It would easy for me to say that we were just lazy but that is never the \emph{whole} truth.  Moreover, Carl and I always have plenty of good stuff to say so it's not like the other prefaces would have been vacuous paragraphs replete with trite aphorisms and buzzwords.  The fact of the matter is that when we started on our open-source adventure back in December 2008, we had no idea what we were getting ourselves into.  We simply wanted to provide a free, high-quality resource to our students without the Imperial Entanglements\footnote{See the Mos Eisley cantina scene in ``Star Wars: Episode IV - A New Hope'' for the inspiration behind this phrase.} of the publishing industry.  As we spent time over the past several summers editing and enhancing the textbook it never occurred to us that a new release of the textbook required a whole new mission statement and thus a new preface.  As a result, we didn't bother to write anything about why we made the changes we made.  In a sense, we viewed the versions which followed our first release more like software updates and less like true editions of a book.

\medskip

\noindent We now find ourselves at a point in the evolution of the Stitz-Zeager Precalculus book where a new mission statement \emph{is} necessary.  To understand why I must take the Preface reader on a journey through the last seven years so I ask that you politely indulge me as I try to tell a complicated story as succinctly as possible.\footnote{Fans of ``The Simpsons'' should think of Season 4, Episode 17 ``Last Exit to Springfield'' and let me know if this story sounds like I have an onion tied to my belt.}  Carl and I have worked on the textbook nearly every summer since we first wrote it.  In fact, we greatly improved the Trigonometry portion during the summer of 2010 (the year after our first sabbatical leaves had ended) because it wasn’t done to our satisfaction in Spring 2010.  In 2011 we were awarded a Gates Foundation Next Generation Learning Challenge grant to edit and enhance the book.  While very well-intentioned, that grant did not allow us to make the kinds of changes we wanted, but rather, focused on creating on-line support resources which never amounted to much of anything useful.  In 2013, we wrote a prerequisites chapter called Chapter 0 to use as part of a pilot ``co-requisite'' section at Lorain County Community College (LCCC). Unfortunately, that section was not as successful as I had hoped it would be and the pilot was discontinued after one semester.  In 2014 my LCCC colleague Heather Bubnick and I were awarded two grants: an Innovative Learning Project Award and an LCCC Foundation Campus Grant to improve and enhance the resources we provide to MTHM 171 (College Algebra) instructors at LCCC.  Those resources are quite good but go unused by many instructors because they are not part of the text.  Meanwhile, Carl kept working to reinvent how he presented some sections of the textbook in his classes and created handouts and lecture notes for his students at Lakeland Community College.\footnote{I will refer to them as ``Lakeland'' as opposed to LCC to avoid confusion between LCCC and LCC.} 

\medskip

\noindent Layered on top of all of the day-to-day work we do as a faculty members at community colleges, Carl and I were both heavily involved at the state level with the Ohio Transfer Module.  We were asked to join a steering committee for the Ohio Mathematics Initiative in 2013 and for the next three years we spent hundreds of hours working on a host of issues.  The fruits of that labor, and much of the labor itself, can be found at the Ohio Department of Higher Education's website \href{https://www.ohiohighered.org/mathematics-initiative}{www.ohiohighered.org/mathematics-initiative}.  Only true policy wonks need to take the time to read all of the material presented there; the rest of you can just continue reading The New Preface.

\noindent 


Talk about co-reqs, reducing dev ed, QR, our work on the OMI.  Provide weblinks to the OMI stuff at ODHE.  Describe the structural renovations.  

Also mention the font change for ADA compliance.  Desmos is replacing the calculator screen shots,  talk about how we define Precalculus to mean College Algebra + College Trigonometry so we've put material from high school (Intermediate Algebra and basic Geometry) in the Appendix.

\noindent  Note about homework:  I was thinking somewhere in our preface we should say we think homework should be a learning experience, so we seed the exercises with problems we hope will really stretch the student. or something.

  If you have any comments, questions or concerns please feel free to contact me at jeff@stitz-zeager.com or Carl at carl@stitz-zeager.com.

\vspace{.5in}

\begin{tabbing}

\hspace{4in}   \= Jeff Zeager \\
               \> Lorain County Community College \\
               \> January 17, 2017
               
\end{tabbing}